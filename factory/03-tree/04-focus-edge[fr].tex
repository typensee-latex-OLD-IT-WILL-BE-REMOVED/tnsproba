\documentclass[12pt,a4paper]{article}

\makeatletter
    \usepackage[utf8]{inputenc}
\usepackage[T1]{fontenc}
\usepackage{ucs}

\usepackage[french]{babel,varioref}

\usepackage[top=2cm, bottom=2cm, left=1.5cm, right=1.5cm]{geometry}
\usepackage{enumitem}

\usepackage{pgffor}

\usepackage{multicol}

\usepackage{makecell}

\usepackage{color}
\usepackage{hyperref}
\hypersetup{
    colorlinks,
    citecolor=black,
    filecolor=black,
    linkcolor=black,
    urlcolor=black
}

\usepackage{amsthm}

\usepackage{tcolorbox}
\tcbuselibrary{listingsutf8}

\usepackage{ifplatform}

\usepackage{ifthen}

\usepackage{macroenvsign}


% Sections numbering

%\renewcommand\thechapter{\Alph{chapter}.}
\renewcommand\thesection{\Roman{section}.}
\renewcommand\thesubsection{\arabic{subsection}.}
\renewcommand\thesubsubsection{\roman{subsubsection}.}



% MISC

\newtcblisting{latexex}{%
	sharp corners,%
	left=1mm, right=1mm,%
	bottom=1mm, top=1mm,%
	colupper=red!75!blue,%
	listing side text
}

\newtcbinputlisting{\inputlatexex}[2][]{%
	listing file={#2},%
	sharp corners,%
	left=1mm, right=1mm,%
	bottom=1mm, top=1mm,%
	colupper=red!75!blue,%
	listing side text
}


\newtcblisting{latexex-flat}{%
	sharp corners,%
	left=1mm, right=1mm,%
	bottom=1mm, top=1mm,%
	colupper=red!75!blue,%
}

\newtcbinputlisting{\inputlatexexflat}[2][]{%
	listing file={#2},%
	sharp corners,%
	left=1mm, right=1mm,%
	bottom=1mm, top=1mm,%
	colupper=red!75!blue,%
}


\newtcblisting{latexex-alone}{%
	sharp corners,%
	left=1mm, right=1mm,%
	bottom=1mm, top=1mm,%
	colupper=red!75!blue,%
	listing only
}

\newtcbinputlisting{\inputlatexexalone}[2][]{%
	listing file={#2},%
	sharp corners,%
	left=1mm, right=1mm,%
	bottom=1mm, top=1mm,%
	colupper=red!75!blue,%
	listing only
}


\newcommand\inputlatexexcodeafter[1]{%
	\begin{center}
		\input{#1}
	\end{center}

	\vspace{-.5em}
	
	Le rendu précédent a été obtenu via le code suivant.
	
	\inputlatexexalone{#1}
}


\newcommand\inputlatexexcodebefore[1]{%
	\inputlatexexalone{#1}
	\vspace{-.75em}
	\begin{center}
		\textit{\footnotesize Rendu du code précédent}
		
		\medskip
		
		\input{#1}
	\end{center}
}


\newcommand\env[1]{\texttt{#1}}
\newcommand\macro[1]{\env{\textbackslash{}#1}}



\setlength{\parindent}{0cm}
\setlist{noitemsep}

\theoremstyle{definition}
\newtheorem*{remark}{Remarque}

\usepackage[raggedright]{titlesec}

\titleformat{\paragraph}[hang]{\normalfont\normalsize\bfseries}{\theparagraph}{1em}{}
\titlespacing*{\paragraph}{0pt}{3.25ex plus 1ex minus .2ex}{0.5em}


\newcommand\separation{
	\medskip
	\hfill\rule{0.5\textwidth}{0.75pt}\hfill
	\medskip
}


\newcommand\extraspace{
	\vspace{0.25em}
}


\newcommand\whyprefix[2]{%
	\textbf{\prefix{#1}}-#2%
}

\newcommand\mwhyprefix[2]{%
	\texttt{#1 = #1-#2}%
}

\newcommand\prefix[1]{%
	\texttt{#1}%
}


\newcommand\inenglish{\@ifstar{\@inenglish@star}{\@inenglish@no@star}}

\newcommand\@inenglish@star[1]{%
	\emph{\og #1 \fg}%
}

\newcommand\@inenglish@no@star[1]{%
	\@inenglish@star{#1} en anglais%
}


\newcommand\ascii{\texttt{ASCII}}


% Example
\newcounter{paraexample}[subsubsection]

\newcommand\@newexample@abstract[2]{%
	\paragraph{%
		#1%
		\if\relax\detokenize{#2}\relax\else {} -- #2\fi%
	}%
}



\newcommand\newparaexample{\@ifstar{\@newparaexample@star}{\@newparaexample@no@star}}

\newcommand\@newparaexample@no@star[1]{%
	\refstepcounter{paraexample}%
	\@newexample@abstract{Exemple \theparaexample}{#1}%
}

\newcommand\@newparaexample@star[1]{%
	\@newexample@abstract{Exemple}{#1}%
}


% Change log
\newcommand\topic{\@ifstar{\@topic@star}{\@topic@no@star}}

\newcommand\@topic@no@star[1]{%
	\textbf{\textsc{#1}.}%
}

\newcommand\@topic@star[1]{%
	\textbf{\textsc{#1} :}%
}



    \usepackage{01-basic}
    \usepackage{04-focus-edge}
    \usepackage{05-auto-num}
\makeatother


% == EXTRAS == %


\begin{document}

%\section{Arbres pondérés}

\subsection{Mettre en valeur des chemins}

\newparaexample{Juste avec deux noeuds}

Il est relativement aisé de mettre en valeur un chemin particulier comme dans l'exemple ci-après qui est une simple démo. montrant les différences entre \macro{ptreeFocus},  \macro{ptreeFocus*} et \macro{ptreeFocus**}.
Notez que les noms des noeuds sont séparés par des barres verticales \verb#|# et  qu'il est possible d'utiliser des espaces pour améliorer la lisibilité du code. 

\inputlatexex{tikz/probatree-focus-edge.tkz}

Voici ce qu'il faut retenir.

\begin{enumerate}
	\item \macro{ptreeFocus} encadre tous les noeuds.

	\item \macro{ptreeFocus*} n'encadre pas le tout premier noeud \emph{(typiquement cela est utile pour un chemin partant de la racine de l'arbre si celle-ci n'est pas nommée comme on le fait très souvent)}.

	\item \macro{ptreeFocus**} n'encadre aucun des noeuds.
	
	\item La couleur peut être changée via l'argument optionnel en utilisant les couleurs de type TikZ. Par défaut le bleu est utilisé.
\end{enumerate}


\begin{remark}
	Le fonctionnement interne de \verb#forest# empêche une coloration automatique des noeuds.
	Si vous souhaitez obtenir cet effet il faudra ajouter les couleurs à la main pour chaque noeud comme dans l'exemple qui suit.

	\inputlatexex{tikz/probatree-focus-edge-color-nodes.tkz}
\end{remark}


% ---------------------- %


\newparaexample{Plusieurs noeuds d'un coup}

Rien de bien compliqué à condition de bien respecter l'ordre de saisie des noeuds.

\inputlatexexflat{tikz/probatree-focus-edge-long-star.tkz}


Avec \macro{ptreeFocus} on obtient l'arbre suivant où le mini disque initial
\footnote{
	Ce disque est en fait un carré aux coins arrondis autour d'un texte vide.
}
n'est pas forcément souhaité.

\input{tikz/probatree-focus-edge-long-no-star.tkz}


Avec \macro{ptreeFocus**} on obtient l'arbre ci-dessous.

\input{tikz/probatree-focus-edge-long-star-star.tkz}


% ---------------------- %


%\newparaexample{De la racine à un noeud}

%Comme il est très courant de vouloir indiquer un chemin de la racine à une feuille, cette fonctionnalité est proposée par le package via les macros \macro{aptreeFocusAll},  \macro{aptreeFocusAll*} et \macro{aptreeFocusAll**} qui prennent comme unique argument obligatoire le noeud final du chemin
%\footnote{
%	Rien n'interdit de prendre un noeud interne au lieu d'u racine même si cela ne semble pas très utile a priori.
%}
%indiqué via le nom automatique fourni par \verb#forest#.
%On peut alors utiliser le code suivant très rapide à taper : voir la section \ref{tnsproba-autonum-forest} page \pageref{tnsproba-autonum-forest} où est expliqué d'où vient le nom \verb#!xxxxx# utilisé ci-après.
%

%\inputlatexexflat{tikz/probatree-focus-edge-all.tkz}
%\splitit{1312}


%\begin{remark}
%	La clé \verb#pfocus# est le 2\iem{} cas particulier de décoration avec \verb#pframe# car les autres décorations se font en dehors de la définition de l'arbre.
%\end{remark}

\end{document}
