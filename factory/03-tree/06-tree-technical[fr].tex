\documentclass[12pt,a4paper]{article}

\makeatletter
    \input{../config/header[fr].sty}
\makeatother


% == EXTRAS == %


\begin{document}

%\section{Arbres pondérés}

\section{Fiches techniques}

\IDenv[n]{probatree}

\IDenv[n]{probatree*}

\IDenv[n]{probatree**}

\Content{} un arbre codé en utilisant la syntaxe supportée par le package \verb#forest#.

\extraspace

\IDkey{pweight}  pour écrire un poids sur le milieu d'une branche.

\IDkey{apweight} pour écrire un poids au-dessus le milieu d'une branche.

\IDkey{bpweight} pour écrire un poids en-dessous du milieu d'une branche.

\extraspace

\IDkey{pweight*} pour indiquer un poids sans l'imprimer.
Dans l'environnement \env{probatree**}, le poids sera affiché comme si on avait utilisé \verb#pweight#.

\IDkey{apweight*} pour indiquer un poids sans l'imprimer.
Dans l'environnement \env{probatree**}, le poids sera affiché comme si on avait utilisé \verb#apweight#.

\IDkey{bpweight*} pour indiquer un poids sans l'imprimer.
Dans l'environnement \env{probatree**}, le poids sera affiché comme si on avait utilisé \verb#bpweight#.

\extraspace

\IDkey{pframe} pour encadrer un sous-arbre depuis un noeud vers toutes les feuilles de celui-ci.


\separation


\IDmacro{ptreeFrame}{1}{3} \hfill \mwhyprefix{p}{robabilty}

\IDoption{} un système de type \texttt{clé=valeur}.

\begin{enumerate}
	\item \verb#col# : une couleur au format TikZ. La valeur par défaut est \verb#blue#.
\end{enumerate}


\IDargs{1..3} les noms de la sous-racine (à gauche), du noeud final en haut (à droite) et du noeud final en bas (à droite) tous indiqués via \verb#name = ...# \emph{(en fait l'ordre n'est pas important ici)}.


\separation


\IDmacro{aptreeFrame}{1}{3} \hfill \mwhyprefix{a}{auto}

\extraspace
\extraspace

Voir les indications précédentes excepté qu'ici on utilise le système de nommage automatisé dérivé de celui de \verb#forest#.


\separation


\IDmacro{ptreeComment}{1}{2}

\IDoption{} un système de type \texttt{clé=valeur}.

\begin{enumerate}
	\item \verb#col# : une couleur au format TikZ. La valeur par défaut est \verb#black#.

	\item \verb#dx# : une distance horizontale relative de décalage. La valeur par défaut est \verb#0cm#.

	\item \verb#dy# : une distance verticale relative de décalage. La valeur par défaut est \verb#0cm#.
\end{enumerate}

\IDarg{1} le nom de la feuille.

\IDarg{2} le texte du commentaire.


\separation


\IDmacro{aptreeComment}{1}{2}

\extraspace
\extraspace

Voir les indications précédentes excepté qu'ici on utilise le système de nommage automatisé dérivé de celui de \verb#forest#.


\separation


\IDmacro{ptreeFocus  }{1}{1}

\IDmacro{ptreeFocus* }{1}{1}

\IDmacro{ptreeFocus**}{1}{1}

\IDoption{} un système de type \texttt{clé=valeur}.

\begin{enumerate}
	\item \verb#col# : une couleur au format TikZ. La valeur par défaut est \verb#blue#.
\end{enumerate}

\IDarg{} les noms des noeuds indiqués via \verb#name = ...# à fournir dans le bon ordre et à séparer par des barres verticales \verb#|#.


\separation


\IDmacro{aptreeFocus  }{1}{1}

\IDmacro{aptreeFocus* }{1}{1}

\IDmacro{aptreeFocus**}{1}{1}

\extraspace
\extraspace

Voir les indications précédentes excepté qu'ici on utilise le système de nommage automatisé dérivé de celui de \verb#forest#.

\end{document}
