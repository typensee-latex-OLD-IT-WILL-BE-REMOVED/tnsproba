\documentclass[12pt,a4paper]{article}

\makeatletter
    \input{../config/header[fr].sty}
    % == PACKAGES USED == %

\RequirePackage{amsmath}


% == DEFINITIONS == %

% Semantic probability

\newcommand\proba[2][p]{%
    \mathchoice{% * Display style
        #1\mskip-.65\medmuskip\left( #2 \right)%
    }{%           * Text style
        #1\mskip-.65\medmuskip\left( #2 \right)%
    }{%           * Script style
        #1\left( #2 \right)%
    }{%           * Script script style
        #1\left( #2 \right)%
    }
}


% Conditional probability

\newcommand\tnsproba@abstract@proba@cond[4]{%
    #1{\proba[#2]{#3 \cap #4}}{\proba[#2]{#4}}%
}


\newcommand\probacond{\@ifstar{\tnsproba@proba@cond@star}{\tnsproba@proba@cond@no@star}}

\newcommand\tnsproba@proba@cond@no@star[3][p]{%
    \proba[#1_{#2}]{#3}%
}

\newcommand\tnsproba@proba@cond@star[3][p]{%
    \proba[#1]{#3 \mid #2}%
}


\newcommand\eprobacond{\@ifstar{\tnsproba@proba@cond@exp@star}{\tnsproba@proba@cond@exp@no@star}}

\newcommand\tnsproba@proba@cond@exp@star[3][p]{%
    \tnsproba@abstract@proba@cond{\frac}{#1}{#3}{#2}
}

\newcommand\tnsproba@proba@cond@exp@no@star[3][p]{%
    \tnsproba@abstract@proba@cond{\dfrac}{#1}{#3}{#2}
}


% "Not" event

\newcommand\nevent[1]{%
    \overline{\kern.15ex#1\vphantom{#1^{x}}\kern.15ex}%
}


% Expected value - Variance - Standard deviation

\newcommand\expval[2][\mathrm{E}]{%
    \proba[#1]{#2}%
}

\newcommand\var[2][\mathrm{V}]{%
    \proba[#1]{#2}%
}

\newcommand\stddev[2][\sigma]{%
    \proba[#1]{#2}%
}



    \usepackage{01-basic}
    \usepackage{02-coment}
    \usepackage{03-tag}
\makeatother


% == EXTRAS == %


\begin{document}

%\section{Arbres pondérés}

\subsection{Décorer les feuilles}

Il peut être utile de décorer de la même façon différents chemins pour indiquer des évènements étudiés.
C'est la raison d'être de \macro{ptreeTagLeaf} et \macro{ptreeTagLeaf*} qui produisent le même commentaire pour différents noeuds.
Notez dans les exemples que les noms des noeuds sont séparés par le symbole \verb+|+ avec la possibilité d'ajouter des espaces pour améliorer la lisibilité du code.


\newparaexample{Décorer au même niveau}

\inputlatexexflat{tikz/probatree-tag-leaf.tkz}


% ---------------------- %


\newparaexample{Décorer sur des niveaux décalés}

La démo. suivante n'utilise que \verb#dx# mais bien entendu \verb#dy# est aussi disponible.

\inputlatexexflat{tikz/probatree-tag-leaf-dx.tkz}


% ---------------------- %


\newparaexample{Décorer au plus près}

Voici un exemple avec un arbre dissymétrique
\footnote{
	Est-ce vraiment pertinent comme usage ?
}.

\inputlatexexflat{tikz/probatree-tag-leaf-star.tkz}


\begin{remark}
	Sans utiliser la version étoilée de \macro{ptreeTagLeaf}, on obtient ce qui suit qui est mauvais.

	\begin{center}
		\input{tikz/probatree-tag-leaf-no-star-bad.tkz}
	\end{center}
	
	Le package \verb#forest# propose \verb#fit = band# qui est très utile dans ce type de situation.
	Voici comment utiliser ceci.
	
	\inputlatexexflat{tikz/probatree-tag-leaf-no-star-good.tkz} 
\end{remark}
 
\end{document}

