\documentclass[12pt,a4paper]{article}

\makeatletter
    \input{../config/header[fr].sty}
    % == PACKAGES USED == %

\RequirePackage{amsmath}


% == DEFINITIONS == %

% Semantic probability

\newcommand\proba[2][p]{%
    \mathchoice{% * Display style
        #1\mskip-.65\medmuskip\left( #2 \right)%
    }{%           * Text style
        #1\mskip-.65\medmuskip\left( #2 \right)%
    }{%           * Script style
        #1\left( #2 \right)%
    }{%           * Script script style
        #1\left( #2 \right)%
    }
}


% Conditional probability

\newcommand\tnsproba@abstract@proba@cond[4]{%
    #1{\proba[#2]{#3 \cap #4}}{\proba[#2]{#4}}%
}


\newcommand\probacond{\@ifstar{\tnsproba@proba@cond@star}{\tnsproba@proba@cond@no@star}}

\newcommand\tnsproba@proba@cond@no@star[3][p]{%
    \proba[#1_{#2}]{#3}%
}

\newcommand\tnsproba@proba@cond@star[3][p]{%
    \proba[#1]{#3 \mid #2}%
}


\newcommand\eprobacond{\@ifstar{\tnsproba@proba@cond@exp@star}{\tnsproba@proba@cond@exp@no@star}}

\newcommand\tnsproba@proba@cond@exp@star[3][p]{%
    \tnsproba@abstract@proba@cond{\frac}{#1}{#3}{#2}
}

\newcommand\tnsproba@proba@cond@exp@no@star[3][p]{%
    \tnsproba@abstract@proba@cond{\dfrac}{#1}{#3}{#2}
}


% "Not" event

\newcommand\nevent[1]{%
    \overline{\kern.15ex#1\vphantom{#1^{x}}\kern.15ex}%
}


% Expected value - Variance - Standard deviation

\newcommand\expval[2][\mathrm{E}]{%
    \proba[#1]{#2}%
}

\newcommand\var[2][\mathrm{V}]{%
    \proba[#1]{#2}%
}

\newcommand\stddev[2][\sigma]{%
    \proba[#1]{#2}%
}



    \usepackage{01-basic}
    \usepackage{02-coment}
    \usepackage{04-textofnodes}
\makeatother


% = =  EXTRAS = =  %


\begin{document}

%\section{Arbres pondérés}

\subsection{Textes des noeuds}

\newparaexample{Changer les couleurs}

On peut a posteriori changer les couleurs du texte et du fond d'un noeud via \macro{ptreeNodeColor} présentée ci-dessous.

\inputlatexexflat{tikz/probatree-txtnode-color.tkz}


Voici ce qu'il faut retenir du code précédent.

\begin{enumerate}
	\item Par défaut le blanc sert de couleur de fond. Ceci se voit dans la mise en forme du noeud $A$ pour lequel \verb#tcol# change juste la couleur du texte qui par défaut sera le noir.
	
	\item Pour changer la couleur de fond, on passe via \verb#bcol#. Ceci permet d'avoir le rendu souhaité pour la mise en forme du noeud $B$
	      \footnote{
	      	Ainsi le fond du noeud et celui du cadre ont la même couleur.
		   }.
	
	\item Les préfixes \prefix{t} et \prefix{b} de \verb#bcol# et \verb#tcol# sont pour \whyprefix{t}{exte} et \whyprefix{b}{ackground}, ce dernier mot signifiant \inenglish{fond}.
	      Quant à \verb#col# c'est pour \whyprefix{col}{or} soit \inenglish{couleur}.
\end{enumerate}



\begin{remark}
	Techniquement toutes les macros présentées dans cette section cachent l'ancien texte d'un noeud par superposition de ce texte en utilisant une couleur identique pour le texte et le fond.  
\end{remark}


% ---------------------- %


\newparaexample{Changer le texte}

On peut a posteriori changer le texte d'un noeud, avec un choix des couleurs, via \macro{ptreeNodeNewText} présentée ci-après.
Noter que les couleurs par défaut du texte et du fond restent le noir et le blanc respectivement.

\inputlatexexflat{tikz/probatree-txtnode-changetxt.tkz}



% ---------------------- %


\newparaexample{Récupérer le texte d'un noeud}

Considérons l'arbre un peu plat suivant.

\inputlatexexflat{tikz/probatree-txtnode-txtafter.tkz}

Une fois ce code inséré il est possible de récupérer \ptreeTextOf{nA} juste en tapant \verb#\ptreeTextOf{nA}#. Voici un exemple concret
\footnote{
	Mais pas forcément pertinent... L'exemple peut être intéressant dans le cadre de contenus produits de façon automatisée.
}
d'utilisation.

\inputlatexexflat{tikz/probatree-txtnode-txtafter-comment.tkz}


\begin{remark}
	Comme la macro \macro{ptreeNodeNewText} utilise le dernier noeud rencontré, il faudra veiller à ne pas vouloir utiliser les textes de noeuds présents dans deux arbres différents et qui ont en même temps le même nom. 
\end{remark}

\end{document}

