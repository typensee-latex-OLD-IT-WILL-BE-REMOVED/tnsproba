\documentclass[12pt,a4paper]{article}

\makeatletter
    \usepackage[utf8]{inputenc}
\usepackage[T1]{fontenc}
\usepackage{ucs}

\usepackage[french]{babel,varioref}

\usepackage[top=2cm, bottom=2cm, left=1.5cm, right=1.5cm]{geometry}
\usepackage{enumitem}

\usepackage{pgffor}

\usepackage{multicol}

\usepackage{makecell}

\usepackage{color}
\usepackage{hyperref}
\hypersetup{
    colorlinks,
    citecolor=black,
    filecolor=black,
    linkcolor=black,
    urlcolor=black
}

\usepackage{amsthm}

\usepackage{tcolorbox}
\tcbuselibrary{listingsutf8}

\usepackage{ifplatform}

\usepackage{ifthen}

\usepackage{macroenvsign}


% Sections numbering

%\renewcommand\thechapter{\Alph{chapter}.}
\renewcommand\thesection{\Roman{section}.}
\renewcommand\thesubsection{\arabic{subsection}.}
\renewcommand\thesubsubsection{\roman{subsubsection}.}



% MISC

\newtcblisting{latexex}{%
	sharp corners,%
	left=1mm, right=1mm,%
	bottom=1mm, top=1mm,%
	colupper=red!75!blue,%
	listing side text
}

\newtcbinputlisting{\inputlatexex}[2][]{%
	listing file={#2},%
	sharp corners,%
	left=1mm, right=1mm,%
	bottom=1mm, top=1mm,%
	colupper=red!75!blue,%
	listing side text
}


\newtcblisting{latexex-flat}{%
	sharp corners,%
	left=1mm, right=1mm,%
	bottom=1mm, top=1mm,%
	colupper=red!75!blue,%
}

\newtcbinputlisting{\inputlatexexflat}[2][]{%
	listing file={#2},%
	sharp corners,%
	left=1mm, right=1mm,%
	bottom=1mm, top=1mm,%
	colupper=red!75!blue,%
}


\newtcblisting{latexex-alone}{%
	sharp corners,%
	left=1mm, right=1mm,%
	bottom=1mm, top=1mm,%
	colupper=red!75!blue,%
	listing only
}

\newtcbinputlisting{\inputlatexexalone}[2][]{%
	listing file={#2},%
	sharp corners,%
	left=1mm, right=1mm,%
	bottom=1mm, top=1mm,%
	colupper=red!75!blue,%
	listing only
}


\newcommand\inputlatexexcodeafter[1]{%
	\begin{center}
		\input{#1}
	\end{center}

	\vspace{-.5em}
	
	Le rendu précédent a été obtenu via le code suivant.
	
	\inputlatexexalone{#1}
}


\newcommand\inputlatexexcodebefore[1]{%
	\inputlatexexalone{#1}
	\vspace{-.75em}
	\begin{center}
		\textit{\footnotesize Rendu du code précédent}
		
		\medskip
		
		\input{#1}
	\end{center}
}


\newcommand\env[1]{\texttt{#1}}
\newcommand\macro[1]{\env{\textbackslash{}#1}}



\setlength{\parindent}{0cm}
\setlist{noitemsep}

\theoremstyle{definition}
\newtheorem*{remark}{Remarque}

\usepackage[raggedright]{titlesec}

\titleformat{\paragraph}[hang]{\normalfont\normalsize\bfseries}{\theparagraph}{1em}{}
\titlespacing*{\paragraph}{0pt}{3.25ex plus 1ex minus .2ex}{0.5em}


\newcommand\separation{
	\medskip
	\hfill\rule{0.5\textwidth}{0.75pt}\hfill
	\medskip
}


\newcommand\extraspace{
	\vspace{0.25em}
}


\newcommand\whyprefix[2]{%
	\textbf{\prefix{#1}}-#2%
}

\newcommand\mwhyprefix[2]{%
	\texttt{#1 = #1-#2}%
}

\newcommand\prefix[1]{%
	\texttt{#1}%
}


\newcommand\inenglish{\@ifstar{\@inenglish@star}{\@inenglish@no@star}}

\newcommand\@inenglish@star[1]{%
	\emph{\og #1 \fg}%
}

\newcommand\@inenglish@no@star[1]{%
	\@inenglish@star{#1} en anglais%
}


\newcommand\ascii{\texttt{ASCII}}


% Example
\newcounter{paraexample}[subsubsection]

\newcommand\@newexample@abstract[2]{%
	\paragraph{%
		#1%
		\if\relax\detokenize{#2}\relax\else {} -- #2\fi%
	}%
}



\newcommand\newparaexample{\@ifstar{\@newparaexample@star}{\@newparaexample@no@star}}

\newcommand\@newparaexample@no@star[1]{%
	\refstepcounter{paraexample}%
	\@newexample@abstract{Exemple \theparaexample}{#1}%
}

\newcommand\@newparaexample@star[1]{%
	\@newexample@abstract{Exemple}{#1}%
}


% Change log
\newcommand\topic{\@ifstar{\@topic@star}{\@topic@no@star}}

\newcommand\@topic@no@star[1]{%
	\textbf{\textsc{#1}.}%
}

\newcommand\@topic@star[1]{%
	\textbf{\textsc{#1} :}%
}


    % == PACKAGES USED == %

\RequirePackage{amsmath}


% == DEFINITIONS == %

% Semantic probability

\newcommand\proba[2][p]{%
    \mathchoice{% * Display style
        #1\mskip-.65\medmuskip\left( #2 \right)%
    }{%           * Text style
        #1\mskip-.65\medmuskip\left( #2 \right)%
    }{%           * Script style
        #1\left( #2 \right)%
    }{%           * Script script style
        #1\left( #2 \right)%
    }
}


% Conditional probability

\newcommand\tnsproba@abstract@proba@cond[4]{%
    #1{\proba[#2]{#3 \cap #4}}{\proba[#2]{#4}}%
}


\newcommand\probacond{\@ifstar{\tnsproba@proba@cond@star}{\tnsproba@proba@cond@no@star}}

\newcommand\tnsproba@proba@cond@no@star[3][p]{%
    \proba[#1_{#2}]{#3}%
}

\newcommand\tnsproba@proba@cond@star[3][p]{%
    \proba[#1]{#3 \mid #2}%
}


\newcommand\eprobacond{\@ifstar{\tnsproba@proba@cond@exp@star}{\tnsproba@proba@cond@exp@no@star}}

\newcommand\tnsproba@proba@cond@exp@star[3][p]{%
    \tnsproba@abstract@proba@cond{\frac}{#1}{#3}{#2}
}

\newcommand\tnsproba@proba@cond@exp@no@star[3][p]{%
    \tnsproba@abstract@proba@cond{\dfrac}{#1}{#3}{#2}
}


% "Not" event

\newcommand\nevent[1]{%
    \overline{\kern.15ex#1\vphantom{#1^{x}}\kern.15ex}%
}


% Expected value - Variance - Standard deviation

\newcommand\expval[2][\mathrm{E}]{%
    \proba[#1]{#2}%
}

\newcommand\var[2][\mathrm{V}]{%
    \proba[#1]{#2}%
}

\newcommand\stddev[2][\sigma]{%
    \proba[#1]{#2}%
}



    \usepackage{01-basic}
    \usepackage{02-coment}
    \usepackage{03-tag-level}
    \usepackage{04-textofnodes}
    \usepackage{05-frame}
    \usepackage{06-focus-edge}
    \usepackage{07-auto-num}
\makeatother


% == EXTRAS == %


\begin{document}

%\section{Arbres pondérés}

\subsection{Utiliser les noms automatiques donnés par \texttt{forest}} \label{tnsproba-autonum-forest}

Voyons comment obtenir le résultat suivant en indiquant tous les noeuds via les noms automatiques fabriqués par \verb@forest@.
%\footnote{
%	Le même rendu peut se taper bien plus efficacement via la macro à un argument \macro{ptreeFocusAll*} comme cela est expliqué dans la section précédente.
%}.

\inputlatexexcodeafter{tikz/probatree-autonum-default.tkz}

% -- DEBUG - TO KEEP ! - START -- %
%\input{tikz/probatree-autonum-start.tkz}
%
%\input{tikz/probatree-autonum-none.tkz}
% -- DEBUG - TO KEEP ! - END -- %



Voici comment s'y prendre.

\begin{enumerate}
	\item On utilise \macro{aptreeFocus} au lieu de \macro{ptreeFocus} où le préfixe \prefix{a} est pour \whyprefix{a}{uto}.

		  \medskip
		  
		  {\itshape
		  		\textbf{ATTENTION !}
		  		Il n'est pas possible de modifier les couleurs du texte et du fond des noeuds car \macro{aptreeTextOf}, \macro{aptreeNodeColor} et \macro{aptreeNodeNewText} n'ont pas pu être implémentées.
		  		Par contre les macros \macro{aptreeComment} et \macro{aptreeFrame} existent sans limitation.
		  }

		  \medskip

	\item Chaque nom automatique
	      \footnote{
	      		Ce sont en fait des noms relativement à la racine de l'arbre.
	      }
	      fabriqué par \verb#forest# commence par \texttt{!} . Ce caractère spécial n'est pas à indiquer car il sera ajouté automatiquement en coulisse.

	
	\item La racine est nommée \texttt{!} par \verb#forest# d'où le \verb#|# seul au début de l'argument de \macro{aptreeFocus*} ci-dessus afin d'indiquer un texte vide comme tout premier noeud.

	
	\item Pour voir ce qu'il faut faire pour un noeud autre que la racine, considérons par exemple \texttt{1321}. On indique en fait le chemin à suivre après la racine pour arriver au noeud voulu.
	\begin{itemize}
		\medskip
		
		\item Aller d'abord au
		      \kern.25em\fbox{1}\kern.4ex\ier{}
		      \kern.15em
		      noeud du niveau 1 qui ici est $A$.

		\item Aller ensuite au
		      \kern.55em\fbox{3}\kern.4ex\ieme{}
		      \kern.4em
		      noeud du niveau 2 qui ici est $D$.

		\item Aller après au
		      \kern1.35em\fbox{2}\kern.4ex\ieme{}
		      \kern.4em
		      noeud du niveau 3 qui ici est $F$.
		
		\item Aller enfin au
		      \kern1.55em\fbox{1}\kern.4ex\ier{}
		      \kern.15em
		      noeud du niveau 4 qui ici est $G$.
		      
		\medskip
		
		\item On obtient finalement notre noeud nommé \fbox{\texttt{1321}}.
	\end{itemize}
\end{enumerate}


\begin{remark}
	Dans la mesure du possible, utiliser les noms automatiques facilite la maintenance des arbres sur le long terme.
	Si on reprend le tout premier exemple d'arbre décoré, il est bien plus simple de faire comme suit car on ne touche pas à la structure minimale du code de l'arbre.
	On a même utilisé \macro{ptreeFrame} au lieu de la clé \verb#pframe# dans l'arbre.

	\inputlatexexflat{tikz/probatree-showcase-autonum.tkz}
\end{remark}


\end{document}
