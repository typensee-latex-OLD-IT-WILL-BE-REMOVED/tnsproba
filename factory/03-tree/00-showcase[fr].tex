\documentclass[12pt,a4paper]{article}

\makeatletter
    \input{../config/header[fr].sty}
    % == PACKAGES USED == %

\RequirePackage{amsmath}


% == DEFINITIONS == %

% Semantic probability

\newcommand\proba[2][p]{%
    \mathchoice{% * Display style
        #1\mskip-.65\medmuskip\left( #2 \right)%
    }{%           * Text style
        #1\mskip-.65\medmuskip\left( #2 \right)%
    }{%           * Script style
        #1\left( #2 \right)%
    }{%           * Script script style
        #1\left( #2 \right)%
    }
}


% Conditional probability

\newcommand\tnsproba@abstract@proba@cond[4]{%
    #1{\proba[#2]{#3 \cap #4}}{\proba[#2]{#4}}%
}


\newcommand\probacond{\@ifstar{\tnsproba@proba@cond@star}{\tnsproba@proba@cond@no@star}}

\newcommand\tnsproba@proba@cond@no@star[3][p]{%
    \proba[#1_{#2}]{#3}%
}

\newcommand\tnsproba@proba@cond@star[3][p]{%
    \proba[#1]{#3 \mid #2}%
}


\newcommand\eprobacond{\@ifstar{\tnsproba@proba@cond@exp@star}{\tnsproba@proba@cond@exp@no@star}}

\newcommand\tnsproba@proba@cond@exp@star[3][p]{%
    \tnsproba@abstract@proba@cond{\frac}{#1}{#3}{#2}
}

\newcommand\tnsproba@proba@cond@exp@no@star[3][p]{%
    \tnsproba@abstract@proba@cond{\dfrac}{#1}{#3}{#2}
}


% "Not" event

\newcommand\nevent[1]{%
    \overline{\kern.15ex#1\vphantom{#1^{x}}\kern.15ex}%
}


% Expected value - Variance - Standard deviation

\newcommand\expval[2][\mathrm{E}]{%
    \proba[#1]{#2}%
}

\newcommand\var[2][\mathrm{V}]{%
    \proba[#1]{#2}%
}

\newcommand\stddev[2][\sigma]{%
    \proba[#1]{#2}%
}



    \usepackage{01-basic}
    \usepackage{02-coment}
    \usepackage{03-tag-level}
    \usepackage{04-textofnodes}
    \usepackage{05-frame}
    \usepackage{06-focus-edge}
    \usepackage{07-auto-num}
\makeatother


% == EXTRAS == %


\begin{document}

\section{Arbres pondérés}

\subsection{Au commencement était la forêt...}

Le gros du travail est fait par le package \verb+forest+ qui s'appuie \verb+TikZ+ dont on peut utiliser toute la machinerie afin d'obtenir des choses sympathiques comme ci-dessous et ceci à moindre coût neuronal comme vont le montrer les explications données dans les sections suivantes.

\inputlatexexcodeafter{TikZ/probatree-showcase.tkz}


\begin{remark}
	Jusqu'à la section \ref{tnsproba-autonum-forest} page \pageref{tnsproba-autonum-forest}, nous nommerons à la main les noeuds des arbres via \verb#name = ...# lorsque cela sera nécessaire.
	Dans la section indiquée nous verrons comment utiliser les noms automatiques donnés par le package \verb#forest#.
\end{remark}

\end{document}
