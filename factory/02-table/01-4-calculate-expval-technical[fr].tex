\documentclass[12pt,a4paper]{article}

\makeatletter
    \usepackage[utf8]{inputenc}
\usepackage[T1]{fontenc}
\usepackage{ucs}

\usepackage[french]{babel,varioref}

\usepackage[top=2cm, bottom=2cm, left=1.5cm, right=1.5cm]{geometry}
\usepackage{enumitem}

\usepackage{pgffor}

\usepackage{multicol}

\usepackage{makecell}

\usepackage{color}
\usepackage{hyperref}
\hypersetup{
    colorlinks,
    citecolor=black,
    filecolor=black,
    linkcolor=black,
    urlcolor=black
}

\usepackage{amsthm}

\usepackage{tcolorbox}
\tcbuselibrary{listingsutf8}

\usepackage{ifplatform}

\usepackage{ifthen}

\usepackage{macroenvsign}


% Sections numbering

%\renewcommand\thechapter{\Alph{chapter}.}
\renewcommand\thesection{\Roman{section}.}
\renewcommand\thesubsection{\arabic{subsection}.}
\renewcommand\thesubsubsection{\roman{subsubsection}.}



% MISC

\newtcblisting{latexex}{%
	sharp corners,%
	left=1mm, right=1mm,%
	bottom=1mm, top=1mm,%
	colupper=red!75!blue,%
	listing side text
}

\newtcbinputlisting{\inputlatexex}[2][]{%
	listing file={#2},%
	sharp corners,%
	left=1mm, right=1mm,%
	bottom=1mm, top=1mm,%
	colupper=red!75!blue,%
	listing side text
}


\newtcblisting{latexex-flat}{%
	sharp corners,%
	left=1mm, right=1mm,%
	bottom=1mm, top=1mm,%
	colupper=red!75!blue,%
}

\newtcbinputlisting{\inputlatexexflat}[2][]{%
	listing file={#2},%
	sharp corners,%
	left=1mm, right=1mm,%
	bottom=1mm, top=1mm,%
	colupper=red!75!blue,%
}


\newtcblisting{latexex-alone}{%
	sharp corners,%
	left=1mm, right=1mm,%
	bottom=1mm, top=1mm,%
	colupper=red!75!blue,%
	listing only
}

\newtcbinputlisting{\inputlatexexalone}[2][]{%
	listing file={#2},%
	sharp corners,%
	left=1mm, right=1mm,%
	bottom=1mm, top=1mm,%
	colupper=red!75!blue,%
	listing only
}


\newcommand\inputlatexexcodeafter[1]{%
	\begin{center}
		\input{#1}
	\end{center}

	\vspace{-.5em}
	
	Le rendu précédent a été obtenu via le code suivant.
	
	\inputlatexexalone{#1}
}


\newcommand\inputlatexexcodebefore[1]{%
	\inputlatexexalone{#1}
	\vspace{-.75em}
	\begin{center}
		\textit{\footnotesize Rendu du code précédent}
		
		\medskip
		
		\input{#1}
	\end{center}
}


\newcommand\env[1]{\texttt{#1}}
\newcommand\macro[1]{\env{\textbackslash{}#1}}



\setlength{\parindent}{0cm}
\setlist{noitemsep}

\theoremstyle{definition}
\newtheorem*{remark}{Remarque}

\usepackage[raggedright]{titlesec}

\titleformat{\paragraph}[hang]{\normalfont\normalsize\bfseries}{\theparagraph}{1em}{}
\titlespacing*{\paragraph}{0pt}{3.25ex plus 1ex minus .2ex}{0.5em}


\newcommand\separation{
	\medskip
	\hfill\rule{0.5\textwidth}{0.75pt}\hfill
	\medskip
}


\newcommand\extraspace{
	\vspace{0.25em}
}


\newcommand\whyprefix[2]{%
	\textbf{\prefix{#1}}-#2%
}

\newcommand\mwhyprefix[2]{%
	\texttt{#1 = #1-#2}%
}

\newcommand\prefix[1]{%
	\texttt{#1}%
}


\newcommand\inenglish{\@ifstar{\@inenglish@star}{\@inenglish@no@star}}

\newcommand\@inenglish@star[1]{%
	\emph{\og #1 \fg}%
}

\newcommand\@inenglish@no@star[1]{%
	\@inenglish@star{#1} en anglais%
}


\newcommand\ascii{\texttt{ASCII}}


% Example
\newcounter{paraexample}[subsubsection]

\newcommand\@newexample@abstract[2]{%
	\paragraph{%
		#1%
		\if\relax\detokenize{#2}\relax\else {} -- #2\fi%
	}%
}



\newcommand\newparaexample{\@ifstar{\@newparaexample@star}{\@newparaexample@no@star}}

\newcommand\@newparaexample@no@star[1]{%
	\refstepcounter{paraexample}%
	\@newexample@abstract{Exemple \theparaexample}{#1}%
}

\newcommand\@newparaexample@star[1]{%
	\@newexample@abstract{Exemple}{#1}%
}


% Change log
\newcommand\topic{\@ifstar{\@topic@star}{\@topic@no@star}}

\newcommand\@topic@no@star[1]{%
	\textbf{\textsc{#1}.}%
}

\newcommand\@topic@star[1]{%
	\textbf{\textsc{#1} :}%
}


\makeatother


% == EXTRAS == %


\begin{document}

%\section{Arbres pondérés}

\section{Fiches techniques}

\IDmacro{calcexpval}{1}{1} \hfill \mwhyprefix{calc}{ulate}

                           \hfill \prefix{expval = exp-ected val-ue}

\IDoption{} un système de type \texttt{clé = valeur}.

\begin{enumerate}
    \item \verb#disp# indique ce qu'il faut afficher. La valeur par défaut est \verb#table#. Voici toutes les valeurs possibles.
	\begin{enumerate}
    	\item \verb#table # : un tableau donnant la loi est affiché seul.

    	\item \verb#all   # : un tableau donnant la loi est affiché suivi du début du calcul de l'espérance.

    	\item \verb#none  # : rien n'est affiché.
	
		\extraspace

    	\item \verb#exp   # : le début du calcul de l'espérance sans le tableau donnant la loi.

    	\item \verb#formal# : la formule avec $\Sigma$ .

    	\item \verb#eval  # : la somme des produits du type $p_k \cdot x_k$ .
	\end{enumerate}


    \medskip

    \item \verb#nosigma# : utilisée seule cette option demande de ne pas afficher la formule avec $\Sigma$ .


    \medskip

    \item \verb#E # : le nom de l'espérance.
		  \hfill
		  Par défaut \verb# E = E# .

    \item \verb#X # : le nom de la variable aléatoire.
		  \hfill
		  Par défaut \verb# X = X# .

    \item \verb#k # : l'indice.
		  \hfill
		  Par défaut \verb# k = k# .

    \item \verb#xk# : la lettre à indicer pour les valeurs de la variable aléatoire.
		  \hfill
		  Par défaut \verb#xk = x# .

    \item \verb#pk# : la lettre à indicer pour les valeurs des probabilités.
		  \hfill
		  Par défaut \verb#pk = p# .


    \medskip

    \item \verb#com# : un texte court éventuel à ajouter entre le tableau donnant la loi et le début du calcul de l'espérance.


    \medskip

    \item \verb#ope# : le symbole de multiplication.
		  \hfill
		  Par défaut \verb#ope = \cdot# .


    \medskip

    \item \verb#name # : un nom éventuel de la loi pour une utilisation ultérieure.

    \item \verb#reuse# : un nom indiquant ce qui doit être réutilisé.


    \medskip

    \item \verb#colors# : le cycle des couleurs au format TikZ séparées par des tirets.
		  
		  \smallskip
		  
		  \hfill
		  Par défaut \verb#colors = red - blue - orange - green!70!black# .
\end{enumerate}
%
%
\IDarg{} définition de la loi de la variable aléatoire finie avec une syntaxe semblable à celle d'un tableau.

\end{document}
