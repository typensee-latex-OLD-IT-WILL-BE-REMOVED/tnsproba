\documentclass[12pt,a4paper]{article}

\makeatletter
    \input{../config/header[fr].sty}
\makeatother


% == EXTRAS == %


\begin{document}

%\section{Arbres pondérés}

\section{Fiches techniques}

\IDmacro{calcexpval}{1}{1} \hfill \mwhyprefix{calc}{ulate}

                           \hfill \prefix{expval = exp-ected val-ue}

\IDoption{} un système de type \texttt{clé = valeur}.

\begin{enumerate}
    \item \verb#disp# indique ce qu'il faut afficher. La valeur par défaut est \verb#table#. Voici toutes les valeurs possibles.
	\begin{enumerate}
    	\item \verb#table # : un tableau donnant la loi est affiché seul.

    	\item \verb#all   # : un tableau donnant la loi est affiché suivi du début du calcul de l'espérance.

    	\item \verb#none  # : rien n'est affiché.
	
		\extraspace

    	\item \verb#exp   # : le début du calcul de l'espérance sans le tableau donnant la loi.

    	\item \verb#formal# : la formule avec $\Sigma$ .

    	\item \verb#eval  # : la somme des produits du type $p_k \cdot x_k$ .
	\end{enumerate}


    \medskip

    \item \verb#nosigma# : utilisée seule cette option demande de ne pas afficher la formule avec $\Sigma$ .


    \medskip

    \item \verb#E # : la macro utilisée pour écrire l'espérance.
		  \hfill
		  Par défaut \verb# E = \expval# .

    \item \verb#X # : le nom de la variable aléatoire.
		  \hfill
		  Par défaut \verb# X = X# .\verb#      #

    \item \verb#k # : l'indice.
		  \hfill
		  Par défaut \verb# k = k# .\verb#      #

    \item \verb#xk# : la lettre à indicer pour les valeurs de la variable aléatoire.
		  \hfill
		  Par défaut \verb#xk = x# .\verb#      #

    \item \verb#pk# : la lettre à indicer pour les valeurs des probabilités.
		  \hfill
		  Par défaut \verb#pk = p# .\verb#      #


    \medskip

    \item \verb#com# : un texte court éventuel à ajouter entre le tableau donnant la loi et le début du calcul de l'espérance.


    \medskip

    \item \verb#ope# : le symbole de multiplication.
		  \hfill
		  Par défaut \verb#ope = \cdot# .


    \medskip

    \item \verb#name # : un nom éventuel de la loi pour une utilisation ultérieure.

    \item \verb#reuse# : un nom indiquant ce qui doit être réutilisé.


    \medskip

    \item \verb#colors# : le cycle des couleurs au format TikZ séparées par des tirets.
		  
		  \smallskip
		  
		  \hfill
		  Par défaut \verb#colors = red - blue - orange - green!70!black# .
\end{enumerate}
%
%
\IDarg{} définition de la loi de la variable aléatoire finie avec une syntaxe semblable à celle d'un tableau.

\end{document}
