\documentclass[12pt,a4paper]{article}

\makeatletter
    \input{../config/header[fr].sty}
    % == PACKAGES USED == %

\RequirePackage{amsmath}


% == DEFINITIONS == %

% Semantic probability

\newcommand\proba[2][p]{%
    \mathchoice{% * Display style
        #1\mskip-.65\medmuskip\left( #2 \right)%
    }{%           * Text style
        #1\mskip-.65\medmuskip\left( #2 \right)%
    }{%           * Script style
        #1\left( #2 \right)%
    }{%           * Script script style
        #1\left( #2 \right)%
    }
}


% Conditional probability

\newcommand\tnsproba@abstract@proba@cond[4]{%
    #1{\proba[#2]{#3 \cap #4}}{\proba[#2]{#4}}%
}


\newcommand\probacond{\@ifstar{\tnsproba@proba@cond@star}{\tnsproba@proba@cond@no@star}}

\newcommand\tnsproba@proba@cond@no@star[3][p]{%
    \proba[#1_{#2}]{#3}%
}

\newcommand\tnsproba@proba@cond@star[3][p]{%
    \proba[#1]{#3 \mid #2}%
}


\newcommand\eprobacond{\@ifstar{\tnsproba@proba@cond@exp@star}{\tnsproba@proba@cond@exp@no@star}}

\newcommand\tnsproba@proba@cond@exp@star[3][p]{%
    \tnsproba@abstract@proba@cond{\frac}{#1}{#3}{#2}
}

\newcommand\tnsproba@proba@cond@exp@no@star[3][p]{%
    \tnsproba@abstract@proba@cond{\dfrac}{#1}{#3}{#2}
}


% "Not" event

\newcommand\nevent[1]{%
    \overline{\kern.15ex#1\vphantom{#1^{x}}\kern.15ex}%
}


% Expected value - Variance - Standard deviation

\newcommand\expval[2][\mathrm{E}]{%
    \proba[#1]{#2}%
}

\newcommand\var[2][\mathrm{V}]{%
    \proba[#1]{#2}%
}

\newcommand\stddev[2][\sigma]{%
    \proba[#1]{#2}%
}



    \usepackage{01-calculate-expval}
\makeatother


% == EXTRAS == %


\begin{document}

%\section{Calculer l'espérance -- Cas fini}

\subsection{Pour un usage différé -- Notations par défaut} \label{tnsproba-calclexpval-name-n-reuse}

Rien de bien compliqué à utiliser comme le montre l'exemple suivant. Notez bien qu'il faut indiquer un argument vide à la macro \macro{calcexpval} lors de la réutilisation.

\begin{latexex-flat}
Voici ma loi de dingue.

\calcexpval[name = maloidedingue]{
    0      & 1   & 2   & 3    & 4    & 5   & 6   \\
    0.2000 & 0.1 & 0.2 & 0.05 & 0.15 & 0.1 & 0.2
}

Je dis des choses, bla, bla...

Et puis finalement j'indique le calcul de l'espérance.

\calcexpval[disp  = exp,
            reuse = maloidedingue]{}
\end{latexex-flat}


\begin{remark}
	Si cela vous convient, il est possible d'utiliser des noms de type \og chemin de fichiers \fg{} comme ci-après
	\footnote{
		Ceci permet d'utiliser des espaces de noms très pratiques à l'usage.
	}.
	
\begin{latexex-flat}
Définissons de façon cachée la loi... %
\calcexpval[disp = none,
            name = ma/loi/de/dingue]{
    0      & 1   & 2   & 3    & 4    & 5   & 6   \\
    0.2000 & 0.1 & 0.2 & 0.05 & 0.15 & 0.1 & 0.2
}%
Bla, bla... Affichons la loi.

\calcexpval[disp  = table,
            reuse = ma/loi/de/dingue]{}
\end{latexex-flat}
\end{remark}


\end{document}
