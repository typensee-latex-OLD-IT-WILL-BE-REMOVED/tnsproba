\documentclass[12pt,a4paper]{article}

\makeatletter
    \input{../config/header[fr].sty}
    % == PACKAGES USED == %

\RequirePackage{amsmath}


% == DEFINITIONS == %

% Semantic probability

\newcommand\proba[2][p]{%
    \mathchoice{% * Display style
        #1\mskip-.65\medmuskip\left( #2 \right)%
    }{%           * Text style
        #1\mskip-.65\medmuskip\left( #2 \right)%
    }{%           * Script style
        #1\left( #2 \right)%
    }{%           * Script script style
        #1\left( #2 \right)%
    }
}


% Conditional probability

\newcommand\tnsproba@abstract@proba@cond[4]{%
    #1{\proba[#2]{#3 \cap #4}}{\proba[#2]{#4}}%
}


\newcommand\probacond{\@ifstar{\tnsproba@proba@cond@star}{\tnsproba@proba@cond@no@star}}

\newcommand\tnsproba@proba@cond@no@star[3][p]{%
    \proba[#1_{#2}]{#3}%
}

\newcommand\tnsproba@proba@cond@star[3][p]{%
    \proba[#1]{#3 \mid #2}%
}


\newcommand\eprobacond{\@ifstar{\tnsproba@proba@cond@exp@star}{\tnsproba@proba@cond@exp@no@star}}

\newcommand\tnsproba@proba@cond@exp@star[3][p]{%
    \tnsproba@abstract@proba@cond{\frac}{#1}{#3}{#2}
}

\newcommand\tnsproba@proba@cond@exp@no@star[3][p]{%
    \tnsproba@abstract@proba@cond{\dfrac}{#1}{#3}{#2}
}


% "Not" event

\newcommand\nevent[1]{%
    \overline{\kern.15ex#1\vphantom{#1^{x}}\kern.15ex}%
}


% Expected value - Variance - Standard deviation

\newcommand\expval[2][\mathrm{E}]{%
    \proba[#1]{#2}%
}

\newcommand\var[2][\mathrm{V}]{%
    \proba[#1]{#2}%
}

\newcommand\stddev[2][\sigma]{%
    \proba[#1]{#2}%
}



    \usepackage{01-calculate-expval}
\makeatother


% == EXTRAS == %


\begin{document}

\section{Calculer l'espérance -- Cas fini}

Il est possible de définir facilement les valeurs d'une variable aléatoire finie (v.a.f.) et d'obtenir si souhaiter le calcul détaillé de son espérance
\footnote{
    Pour l'affichage une limitation importante est que le tableau et les calculs doivent tenir sur la largeur de la ligne.
    Concrètement pour un usage pédagogique cette limitation ne devrait jamais poser de problème.
}.


% ---------------------- %


\subsection{Pour un usage direct -- Notations par défaut}

Dans la section \ref{tnsproba-calclexpval-name-n-reuse} page \pageref{tnsproba-calclexpval-name-n-reuse} nous verrons qu'il est possible de définir une loi à un endroit puis de la réutiliser à d'autres
\footnote{
    La réutilisation se fera avec un argument vide de \macro{calcexpval}. C'est pour cela qu'une macro est proposée et non un environnement.
}.
Nous verrons aussi dans la section \ref{tnsproba-calclexpval-notations} page \pageref{tnsproba-calclexpval-notations} que certaines notations sont modifiables.

\medskip

Dans la présente section nous reproduirons à chaque fois les paramètres de la v.a.f. même si ceci produit des exemples un peu lourds à lire.


% ---------------------- %


\newparaexample{Loi d'une v.a.f.}

La définition de la loi d'une v.a.f. avec la macro \macro{calcexpval} se fait avec une syntaxe semblable à celle d'un tableau.
Notez dans l'exemple suivant que par défaut un tableau centré est affiché ce qui correspond à l'option par  défaut \verb#disp = table#.
Ce n'est pas le seul comportement possible comme vont le montrer les autres exemples à venir.

\begin{latexex-flat}
\calcexpval{
    0      & 1   & 2   & 3    & 4    & 5   & 6   \\
    0.2000 & 0.1 & 0.2 & 0.05 & 0.15 & 0.1 & 0.2
}
\end{latexex-flat}


\begin{remark}
    Il est possible de ne rien afficher comme ci-dessous où \prefix{disp} est pour \whyprefix{disp}{lay} soit \inenglish{afficher}.
    Cette option sera utile pour définir une loi à un endroit et l'utiliser à d'autres.

\begin{latexex-flat}
\calcexpval[disp = none]{
    0      & 1   & 2   & 3    & 4    & 5   & 6   \\
    0.2000 & 0.1 & 0.2 & 0.05 & 0.15 & 0.1 & 0.2
}
\end{latexex-flat}
\end{remark}


% ---------------------- %


\newparaexample{Calculs expliqués et décorés de l'espérance d'une v.a.f.}

La macro \macro{calcexpval} propose un système de clé valeur pour des réglages personnalisé. L'exemple ci-dessous montre comment obtenir un calcul détaillé et décoré de l'espérance.

\begin{latexex-flat}
\calcexpval[disp = all]{
    0      & 1   & 2   & 3    & 4    & 5   & 6   \\
    0.2000 & 0.1 & 0.2 & 0.05 & 0.15 & 0.1 & 0.2
}
\end{latexex-flat}


L'exemple suivant montre qu'il est facile de changer le couleurs qui sont utilisées de façon cyclique \emph{(on peut indiquer une seule couleur)}.

\begin{latexex-flat}
\calcexpval[disp   = all,
            colors = orange - olive]{
    0      & 1   & 2   & 3    & 4    & 5   & 6   \\
    0.2000 & 0.1 & 0.2 & 0.05 & 0.15 & 0.1 & 0.2
}
\end{latexex-flat}


% ---------------------- %


\newparaexample{Juste détailler les calculs de l'espérance d'une v.a.f.}

Ci-dessous \prefix{exp} est pour \whyprefix{exp}{and} soit \inenglish{développer}. Cette option sera très utile de la réutilisation d'une loi définie précédemment.

\begin{latexex-flat}
\calcexpval[disp = exp]{
    0      & 1   & 2   & 3    & 4    & 5   & 6   \\
    0.2000 & 0.1 & 0.2 & 0.05 & 0.15 & 0.1 & 0.2
}
\end{latexex-flat}


% ---------------------- %


\newparaexample{Des bouts du calcul}

Lors de la réutilisation d'une loi définie précédemment les deux options \prefix{formal} et \prefix{eval} peuvent rendre service.

\begin{latexex-flat}
Somme formelle :
\calcexpval[disp = formal]{
    0      & 1   & 2   & 3    & 4    & 5   & 6   \\
    0.2000 & 0.1 & 0.2 & 0.05 & 0.15 & 0.1 & 0.2
}

Somme développée :
\calcexpval[disp = eval]{
    0      & 1   & 2   & 3    & 4    & 5   & 6   \\
    0.2000 & 0.1 & 0.2 & 0.05 & 0.15 & 0.1 & 0.2
}

\end{latexex-flat}


% ---------------------- %


\newparaexample{Calculs expliqués sans $\Sigma$}

L'option \verb#nosigma# permet de ne pas afficher le symbole $\Sigma$ pour des raisons pédagogiques ou d'efficacité.

\begin{latexex-flat}
\calcexpval[disp = all,
            nosigma]{
    0      & 1   & 2   & 3    & 4    & 5   & 6   \\
    0.2000 & 0.1 & 0.2 & 0.05 & 0.15 & 0.1 & 0.2
}

\medskip

Juste les calculs :
\calcexpval[disp = exp,
            nosigma]{
    0      & 1   & 2   & 3    & 4    & 5   & 6   \\
    0.2000 & 0.1 & 0.2 & 0.05 & 0.15 & 0.1 & 0.2
} .
\end{latexex-flat}

\end{document}
