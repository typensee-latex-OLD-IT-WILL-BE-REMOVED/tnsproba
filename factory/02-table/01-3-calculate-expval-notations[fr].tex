\documentclass[12pt,a4paper]{article}

\makeatletter
    \input{../config/header[fr].sty}
    % == PACKAGES USED == %

\RequirePackage{amsmath}


% == DEFINITIONS == %

% Semantic probability

\newcommand\proba[2][p]{%
    \mathchoice{% * Display style
        #1\mskip-.65\medmuskip\left( #2 \right)%
    }{%           * Text style
        #1\mskip-.65\medmuskip\left( #2 \right)%
    }{%           * Script style
        #1\left( #2 \right)%
    }{%           * Script script style
        #1\left( #2 \right)%
    }
}


% Conditional probability

\newcommand\tnsproba@abstract@proba@cond[4]{%
    #1{\proba[#2]{#3 \cap #4}}{\proba[#2]{#4}}%
}


\newcommand\probacond{\@ifstar{\tnsproba@proba@cond@star}{\tnsproba@proba@cond@no@star}}

\newcommand\tnsproba@proba@cond@no@star[3][p]{%
    \proba[#1_{#2}]{#3}%
}

\newcommand\tnsproba@proba@cond@star[3][p]{%
    \proba[#1]{#3 \mid #2}%
}


\newcommand\eprobacond{\@ifstar{\tnsproba@proba@cond@exp@star}{\tnsproba@proba@cond@exp@no@star}}

\newcommand\tnsproba@proba@cond@exp@star[3][p]{%
    \tnsproba@abstract@proba@cond{\frac}{#1}{#3}{#2}
}

\newcommand\tnsproba@proba@cond@exp@no@star[3][p]{%
    \tnsproba@abstract@proba@cond{\dfrac}{#1}{#3}{#2}
}


% "Not" event

\newcommand\nevent[1]{%
    \overline{\kern.15ex#1\vphantom{#1^{x}}\kern.15ex}%
}


% Expected value - Variance - Standard deviation

\newcommand\expval[2][\mathrm{E}]{%
    \proba[#1]{#2}%
}

\newcommand\var[2][\mathrm{V}]{%
    \proba[#1]{#2}%
}

\newcommand\stddev[2][\sigma]{%
    \proba[#1]{#2}%
}



    \usepackage{01-calculate-expval}
\makeatother


% == EXTRAS == %


\begin{document}

%\section{Calculer l'espérance -- Cas fini}

\subsection{Notations personnalisées} \label{tnsproba-calclexpval-notations}

\newparaexample{Toutes les options}

L'exemple ci-après utilise toutes les options liées à l'affichage textuel.
Il faut noter que la clé \verb#com# sert juste à ajouter un court texte sur une seule ligne
\footnote{
	Si besoin utiliser séparément l'affichage par défaut avec \texttt{name = maloi} suivi d'un long texte puis enfin employer \texttt{disp = exp, reuse = maloi}.
}.

\inputlatexexflat{examples/expval-use-notation-basic.extra.tex}


% ---------------------- %


\newparaexample{En réutilisant une loi définie précédemment}

Lors de la création d'une loi nommée via la clé \verb#name# les options textuelles, autre que le commentaire court, sont mémorisées.
Ceci est très utile comme le montre le 1\ier{} affichage de l'exemple suivant tout en étant modifiable à tout moment comme dans le dernier affichage ci-après.

\inputlatexexflat{examples/expval-use-notation-reuse.extra.tex}

\end{document}
