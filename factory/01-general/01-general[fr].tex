\documentclass[12pt,a4paper]{article}

\makeatletter
	\input{../config/header[fr].sty}

	\usepackage{01-general}
\makeatother



\begin{document}

\section{Juste pour rédiger}

\subsection{Probabilité \og simple \fg}

\newparaexample{}

\begin{latexex}
$\proba{A}$
\end{latexex}


% ---------------------- %


\newparaexample{Choisir le nom de la probabilité}

\begin{latexex}
$\proba[P]{A}$
\end{latexex}


% ---------------------- %


\subsection{Fiches techniques}

\IDmacro{proba}{1}{1}

\IDoption{} le nom de la probabilité. La valeur par défaut est $p$.

\IDarg{} l'ensemble dont on veut calculer la probabilité.


% ---------------------- %


\subsection{Probabilité conditionnelle}

\newparaexample{Les deux écritures classiques}

La 1\iere{} notation, qui est devenue standard, permet de comprendre l'ordre des arguments.
\begin{latexex}
 $\probacond {B}{A}
= \probacond*{B}{A}$
\end{latexex}


% ---------------------- %


\newparaexample{Obtenir la formule de définition}

Le préfixe \prefix{e} est pour \whyprefix{e}{xpand} soit \inenglish{développer}
\footnote{
	Pour ne pas alourdir l'utilisation de \macro{probacond}, il a été choisi d'utiliser un préfixe au lieu d'un système de multi-options.
}.

\begin{latexex}
 $\eprobacond {B}{A}
= \eprobacond*{B}{A}$
\end{latexex}


% ---------------------- %


\newparaexample{Choisir le nom de la probabilité}

\begin{latexex}
 $\probacond  [P]{B}{A}
= \probacond* [P]{B}{A}
= \eprobacond*[P]{B}{A}
= \eprobacond [P]{B}{A}$
\end{latexex}


% ---------------------- %


\subsection{Fiches techniques}

\IDmacro{probacond  }{1}{2}

\IDmacro{probacond* }{1}{2}

\extraspace

\IDmacro{eprobacond }{1}{2}

\IDmacro{eprobacond*}{1}{2}


\IDoption{} le nom de la probabilité. La valeur par défaut est $p$.

\IDarg{1} l'ensemble qui donne la condition.

\IDarg{2} l'ensemble dont on veut calculer la probabilité.


\end{document}
