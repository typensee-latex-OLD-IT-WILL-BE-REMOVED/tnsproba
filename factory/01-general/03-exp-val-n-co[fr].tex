\documentclass[12pt,a4paper]{article}

\makeatletter
	\input{../config/header[fr].sty}

	\usepackage{01-general}
\makeatother



\begin{document}

%\section{Généralités}

\subsection{Espérance, variance et écart-type}

\newparaexample{Espérance}

\macro{expval} vient de \whyprefix{exp}{ected} \whyprefix{val}{ue} soit \inenglish{espérance}.
\begin{latexex}
$\expval{X}$
\end{latexex}


% ---------------------- %


\newparaexample{Choisir le nom de l'espérance}

\begin{latexex}
$\expval[E_1]{X}$
\end{latexex}


% ---------------------- %


\newparaexample{Variance}

Notez la possibilité d'utiliser \macro{mathit} pour obtenir des lettres en italique.

\begin{latexex}
$\var   {X}$ ou
$\var[v]{X}$ ou
$\var[\mathit{v}]{X}$
\end{latexex}


% ---------------------- %


\newparaexample{Écart-type}

\macro{stddev} vient de \whyprefix{st}{andar-}\prefix{d} \whyprefix{dev}{iation} soit \inenglish{écart-type}.

\begin{latexex}
$\stddev   {X}$ ou
$\stddev[s]{X}$ ou
$\stddev[\mathit{s}]{X}$
\end{latexex}


% ---------------------- %


\subsection{Fiches techniques}

\IDmacro{expval}{1}{1}

\IDoption{} le nom de la fonction espérance. valeur par défaut est \verb#E#.

\IDarg{} la variable aléatoire dont on veut calculer l'espérance.


\separation


\IDmacro{var}{1}{1}

\IDoption{} le nom de la fonction variance. valeur par défaut est \verb#V#.

\IDarg{} la variable aléatoire dont on veut calculer la variance.


\separation


\IDmacro{stddev}{1}{1}

\IDoption{} le nom de la fonction écart-type. La valeur par défaut est \verb#\sigma#.

\IDarg{} la variable aléatoire dont on veut calculer l'écart-type.

\end{document}
