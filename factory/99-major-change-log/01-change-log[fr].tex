\documentclass[12pt,a4paper]{article}

\makeatletter
    \input{../config/header[fr].sty}
\makeatother


\begin{document}

\newpage

\section{Historique}

Nous ne donnons ici qu'un très bref historique récent
\footnote{
	On ne va pas au-delà de un an depuis la dernière version.
}
de \verb+tnsproba+ à destination de l'utilisateur principalement.
Tous les changements sont disponibles uniquement en anglais dans le dossier \verb+change-log+ : voir le code source de \verb+tnsproba+ sur \verb+github+.

\begin{description}
% Changes shown - START

    \medskip
    \item[2020-09-02] Nouvelle version mineure \verb+0.9.0-beta+.
    
    \begin{itemize}[itemsep=.5em]
        \item \topic*{Calculs de l'espérance dans le cas fini}
        	  \macro{calcexpval} rend facile la définition d'une variable aléatoire finie avec la possibilité de détailler le calcul de son espérance.
    
    
    % ---------------------- %
    
    
        \item \topic*{Arbre de probabilités}
        	  pas mal de changements dans l'API et quelques nouveautés.
    
        \begin{itemize}[itemsep=.5em]
            \item \verb#\begin{probatree}<hideall># \verb#...# \verb#\end{probatree}#  remplace l'usage de l'environnement \env{probatree*}.
    
    
            \item \verb#\begin{probatree}<showall># \verb#...# \verb#\end{probatree}# remplace l'usage de l'environnement \env{probatree**}.
    
    
            \item De nouvelles macros permettent d'agir sur le texte des noeuds.
    
            \begin{enumerate}
            	\item \macro{ptreeTextOf} renvoie le texte d'un noeud.
    
            	\item \macro{ptreeNodeColor} change les couleurs du texte et/ou du fond d'un noeud.
    
            	\item \macro{ptreeNodeNewText} change le texte en plus éventuellement  les couleurs du texte et/ou du fond d'un noeud.
            \end{enumerate}
    
    
            \item \macro{ptreeFocus} a évolué
                  \footnote{
                  		Malheureusement ces changements n'ont pas pu être faits pour \macro{aptreeFocus} principalement car l'équivalent automatique de \macro{ptreeTextOf} n'a pu être implémenté.
    			  }.
    
            \begin{enumerate}
            	\item \macro{ptreeFocus[frame = start]} remplace l'ancien \macro{ptreeFocus}.
    
            	\item \macro{ptreeFocus[frame = none]} remplace \macro{ptreeFocus**} qui n'existe plus.
    
            	\item \macro{ptreeFocus[frame = nostart]} est utilisé par défaut et remplace \macro{ptreeFocus*} qui n'existe plus. On obtient dans ce cas une mise en valeur encadrant tous les noeuds sauf le tout 1\ier{}.
    
            	\item Les clés optionnelles \verb#lcol#, \verb#tcol# et \verb#bcol# permettent de choisir la couleur des arrêtes et des cadres, celle du texte et enfin celle du fond.
            \end{enumerate}
    
    
            \item \verb#tcol# remplace l'ancien \verb#col# de \macro{ptreeComment} et \macro{aptreeComment}.
    
    		\item \verb#lcol# remplace l'ancien \verb#col# de \macro{ptreeFrame}, \macro{aptreeFrame} et \macro{aptreeFocus}.
    
    
    
        \end{itemize}
    \end{itemize}
    
    
    \separation

% ------------------------ %

    \medskip
    \item[2020-08-10] Nouvelle version mineure \verb+0.8.0-beta+.
    
    \begin{itemize}[itemsep=.5em]
        \item \topic*{Arbre de probabilités}
              le mode mathématique est activé par défaut pour les noms des noeuds et les poids \emph{(plus besoin de taper plein de \texttt{\$})}.
    \end{itemize}
    
    
    \separation
% ------------------------ %

    \medskip
    \item[2020-08-09] Nouvelle version mineure \verb+0.7.0-beta+.
    
    \begin{itemize}[itemsep=.5em]
        \item \topic{Arbre de probabilités}
        \begin{itemize}[itemsep=.5em]
            \item Utilisation obligatoire de \verb#col=...# pour indiquer une couleur à toutes les macros de décoration.
            
            \item Les macros \macro{ptreeComment}, \macro{ptreeComment*}, 
                  \macro{aptreeComment} et \macro{aptreeComment*}
                  ont deux clés \verb#dx# et \verb#dy# pour indiquer un décalage relatif.
            
            \item Ajout de l'environnement \env{probatree**} qui force l'affichage de tous les poids !
        \end{itemize}
    \end{itemize}
    
    
    \separation
% ------------------------ %

    \medskip
    \item[2020-08-08] Nouvelle version mineure \verb+0.6.0-beta+.
    
    \begin{itemize}[itemsep=.5em]
        \item \topic{Arbre de probabilités}
        \begin{itemize}[itemsep=.5em]
            \item \macro{aptreeFocus}, \macro{aptreeFocus*} et \macro{aptreeFocus**} permettent d'utiliser le système de nommage automatique des noeuds proposé par \verb#forest#.
            
            \item Il en va de même pour \macro{aptreeComment} et \macro{aptreeFrame}.
        \end{itemize}
    \end{itemize}
    
    
    \separation
% ------------------------ %

    \medskip
    \item[2020-08-05] Nouvelle version mineure \verb+0.5.0-beta+.
    
    \begin{itemize}[itemsep=.5em]
        \item \topic{Arbre de probabilités}
        \begin{itemize}[itemsep=.5em]
            \item \macro{ptreeFocus}, \macro{ptreeFocus*} et \macro{ptreeFocus**} fonctionnent avec un multi-argument pour pourvoir indiquer un chemin sur plusieurs noeuds.
            
            \item Suppression de la clé \macro{pcomment}.
            
            \item Ajout des macros \macro{ptreeComment} et \macro{ptreeComment*} qui simplifient la saisie.
        \end{itemize}
    \end{itemize}
    
    \separation

% ------------------------ %

    \medskip
    \item[2020-07-31] Nouvelle version mineure \verb+0.4.0-beta+.
    
    \begin{itemize}[itemsep=.5em]
        \item \topic*{Arbre}
        	  possibilité de mettre en valeur un chemin via \macro{ptreeFocus},  \macro{ptreeFocus*} ou \macro{ptreeFocus**}.
    \end{itemize}
    
    \separation
% ------------------------ %

    \medskip
    \item[2020-07-25] Nouvelle version mineure \verb+0.3.0-beta+.
    
    \begin{itemize}[itemsep=.5em]
        \item \topic{Arbre}
        \begin{itemize}[itemsep=.5em]
            \item Ajout du style \prefix{pcomment} pour placer du texte à la droite d'une feuille.
    
            \item Le style \prefix{frame} a été renommé \prefix{pframe}.
        \end{itemize}
    \end{itemize}
    
    \separation
% ------------------------ %

    \medskip
    \item[2020-07-23] Nouvelle version mineure \verb+0.2.0-beta+.
    
    \begin{itemize}[itemsep=.5em]
        \item \topic*{Arbre}
              ajout de la macro \macro{ptreeFrame} pour tracer facilement des sous cadres non \og finaux \fg.
    \end{itemize}
    
    \separation
% ------------------------ %

    \medskip
    \item[2020-07-22] Nouvelle version mineure \verb+0.1.0-beta+.
    
    \begin{itemize}[itemsep=.5em]
        \item \topic*{Probabilité conditionnelle}
              \macro{probacondexp} renommée en  \macro{eprobacond}.
    
        \item \topic*{Évènement contraire}
              ajout de \macro{nevent}.
    
        \item \topic*{variance et écart-type}
              ajout de \macro{var} et \macro{stddev}.
    \end{itemize}
    
    \separation
% ------------------------ %

    \medskip
    \item[2020-07-10] Première version \verb+0.0.0-beta+.
% ------------------------ %

% Changes shown - END 
\end{description}

\end{document}
