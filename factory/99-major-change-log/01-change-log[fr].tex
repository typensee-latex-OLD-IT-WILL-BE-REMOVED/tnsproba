\documentclass[12pt,a4paper]{article}

\makeatletter
    \usepackage[utf8]{inputenc}
\usepackage[T1]{fontenc}
\usepackage{ucs}

\usepackage[french]{babel,varioref}

\usepackage[top=2cm, bottom=2cm, left=1.5cm, right=1.5cm]{geometry}
\usepackage{enumitem}

\usepackage{pgffor}

\usepackage{multicol}

\usepackage{makecell}

\usepackage{color}
\usepackage{hyperref}
\hypersetup{
    colorlinks,
    citecolor=black,
    filecolor=black,
    linkcolor=black,
    urlcolor=black
}

\usepackage{amsthm}

\usepackage{tcolorbox}
\tcbuselibrary{listingsutf8}

\usepackage{ifplatform}

\usepackage{ifthen}

\usepackage{macroenvsign}


% Sections numbering

%\renewcommand\thechapter{\Alph{chapter}.}
\renewcommand\thesection{\Roman{section}.}
\renewcommand\thesubsection{\arabic{subsection}.}
\renewcommand\thesubsubsection{\roman{subsubsection}.}



% MISC

\newtcblisting{latexex}{%
	sharp corners,%
	left=1mm, right=1mm,%
	bottom=1mm, top=1mm,%
	colupper=red!75!blue,%
	listing side text
}

\newtcbinputlisting{\inputlatexex}[2][]{%
	listing file={#2},%
	sharp corners,%
	left=1mm, right=1mm,%
	bottom=1mm, top=1mm,%
	colupper=red!75!blue,%
	listing side text
}


\newtcblisting{latexex-flat}{%
	sharp corners,%
	left=1mm, right=1mm,%
	bottom=1mm, top=1mm,%
	colupper=red!75!blue,%
}

\newtcbinputlisting{\inputlatexexflat}[2][]{%
	listing file={#2},%
	sharp corners,%
	left=1mm, right=1mm,%
	bottom=1mm, top=1mm,%
	colupper=red!75!blue,%
}


\newtcblisting{latexex-alone}{%
	sharp corners,%
	left=1mm, right=1mm,%
	bottom=1mm, top=1mm,%
	colupper=red!75!blue,%
	listing only
}

\newtcbinputlisting{\inputlatexexalone}[2][]{%
	listing file={#2},%
	sharp corners,%
	left=1mm, right=1mm,%
	bottom=1mm, top=1mm,%
	colupper=red!75!blue,%
	listing only
}


\newcommand\inputlatexexcodeafter[1]{%
	\begin{center}
		\input{#1}
	\end{center}

	\vspace{-.5em}
	
	Le rendu précédent a été obtenu via le code suivant.
	
	\inputlatexexalone{#1}
}


\newcommand\inputlatexexcodebefore[1]{%
	\inputlatexexalone{#1}
	\vspace{-.75em}
	\begin{center}
		\textit{\footnotesize Rendu du code précédent}
		
		\medskip
		
		\input{#1}
	\end{center}
}


\newcommand\env[1]{\texttt{#1}}
\newcommand\macro[1]{\env{\textbackslash{}#1}}



\setlength{\parindent}{0cm}
\setlist{noitemsep}

\theoremstyle{definition}
\newtheorem*{remark}{Remarque}

\usepackage[raggedright]{titlesec}

\titleformat{\paragraph}[hang]{\normalfont\normalsize\bfseries}{\theparagraph}{1em}{}
\titlespacing*{\paragraph}{0pt}{3.25ex plus 1ex minus .2ex}{0.5em}


\newcommand\separation{
	\medskip
	\hfill\rule{0.5\textwidth}{0.75pt}\hfill
	\medskip
}


\newcommand\extraspace{
	\vspace{0.25em}
}


\newcommand\whyprefix[2]{%
	\textbf{\prefix{#1}}-#2%
}

\newcommand\mwhyprefix[2]{%
	\texttt{#1 = #1-#2}%
}

\newcommand\prefix[1]{%
	\texttt{#1}%
}


\newcommand\inenglish{\@ifstar{\@inenglish@star}{\@inenglish@no@star}}

\newcommand\@inenglish@star[1]{%
	\emph{\og #1 \fg}%
}

\newcommand\@inenglish@no@star[1]{%
	\@inenglish@star{#1} en anglais%
}


\newcommand\ascii{\texttt{ASCII}}


% Example
\newcounter{paraexample}[subsubsection]

\newcommand\@newexample@abstract[2]{%
	\paragraph{%
		#1%
		\if\relax\detokenize{#2}\relax\else {} -- #2\fi%
	}%
}



\newcommand\newparaexample{\@ifstar{\@newparaexample@star}{\@newparaexample@no@star}}

\newcommand\@newparaexample@no@star[1]{%
	\refstepcounter{paraexample}%
	\@newexample@abstract{Exemple \theparaexample}{#1}%
}

\newcommand\@newparaexample@star[1]{%
	\@newexample@abstract{Exemple}{#1}%
}


% Change log
\newcommand\topic{\@ifstar{\@topic@star}{\@topic@no@star}}

\newcommand\@topic@no@star[1]{%
	\textbf{\textsc{#1}.}%
}

\newcommand\@topic@star[1]{%
	\textbf{\textsc{#1} :}%
}


\makeatother


\begin{document}

\newpage

\section{Historique}

Nous ne donnons ici qu'un très bref historique récent
\footnote{
	On ne va pas au-delà de un an depuis la dernière version.
}
de \verb+tnsproba+ à destination de l'utilisateur principalement.
Tous les changements sont disponibles uniquement en anglais dans le dossier \verb+change-log+ : voir le code source de \verb+tnsproba+ sur \verb+github+.

\begin{description}
% Changes shown - START

    \medskip
    \item[2020-08-09] Nouvelle version mineure \verb+0.7.0-beta+.
    
    \begin{itemize}[itemsep=.5em]
        \item \topic{Arbre de probabilités}
        \begin{itemize}[itemsep=.5em]
            \item Utilisation obligatoire de \verb#col=...# pour indiquer une couleur à toutes les macros de décoration.
            
            \item Les macros \macro{ptreeComment}, \macro{ptreeComment*}, 
                  \macro{aptreeComment} et \macro{aptreeComment*}
                  ont deux clés \verb#dx# et \verb#dy# pour indiquer un décalage relatif.
            
            \item Ajout de l'environnement \env{probatree**} qui force l'affichage de tous les poids !
        \end{itemize}
    \end{itemize}
    
    
    \separation
% ------------------------ %

    \medskip
    \item[2020-08-08] Nouvelle version mineure \verb+0.6.0-beta+.
    
    \begin{itemize}[itemsep=.5em]
        \item \topic{Arbre de probabilités}
        \begin{itemize}[itemsep=.5em]
            \item \macro{aptreeFocus}, \macro{aptreeFocus*} et \macro{aptreeFocus**} permettent d'utiliser le système de nommage automatique des noeuds proposé par \verb#forest#.
            
            \item Il en va de même pour \macro{aptreeComment} et \macro{aptreeFrame}.
        \end{itemize}
    \end{itemize}
    
    
    \separation
% ------------------------ %

    \medskip
    \item[2020-08-05] Nouvelle version mineure \verb+0.5.0-beta+.
    
    \begin{itemize}[itemsep=.5em]
        \item \topic{Arbre de probabilités}
        \begin{itemize}[itemsep=.5em]
            \item \macro{ptreeFocus}, \macro{ptreeFocus*} et \macro{ptreeFocus**} fonctionnent avec un multi-argument pour pourvoir indiquer un chemin sur plusieurs noeuds.
            
            \item Suppression de la clé \macro{pcomment}.
            
            \item Ajout des macros \macro{ptreeComment} et \macro{ptreeComment*} qui simplifient la saisie.
        \end{itemize}
    \end{itemize}
    
    \separation

% ------------------------ %

    \medskip
    \item[2020-07-31] Nouvelle version mineure \verb+0.4.0-beta+.
    
    \begin{itemize}[itemsep=.5em]
        \item \topic*{Arbre}
        	  possibilité de mettre en valeur un chemin via \macro{ptreeFocus},  \macro{ptreeFocus*} ou \macro{ptreeFocus**}.
    \end{itemize}
    
    \separation
% ------------------------ %

    \medskip
    \item[2020-07-25] Nouvelle version mineure \verb+0.3.0-beta+.
    
    \begin{itemize}[itemsep=.5em]
        \item \topic{Arbre}
        \begin{itemize}[itemsep=.5em]
            \item Ajout du style \prefix{pcomment} pour placer du texte à la droite d'une feuille.
    
            \item Le style \prefix{frame} a été renommé \prefix{pframe}.
        \end{itemize}
    \end{itemize}
    
    \separation
% ------------------------ %

    \medskip
    \item[2020-07-23] Nouvelle version mineure \verb+0.2.0-beta+.
    
    \begin{itemize}[itemsep=.5em]
        \item \topic*{Arbre}
              ajout de la macro \macro{ptreeFrame} pour tracer facilement des sous cadres non \og finaux \fg.
    \end{itemize}
    
    \separation
% ------------------------ %

    \medskip
    \item[2020-07-22] Nouvelle version mineure \verb+0.1.0-beta+.
    
    \begin{itemize}[itemsep=.5em]
        \item \topic*{Probabilité conditionnelle}
              \macro{probacondexp} renommée en  \macro{eprobacond}.
    
        \item \topic*{Évènement contraire}
              ajout de \macro{nevent}.
    
        \item \topic*{variance et écart-type}
              ajout de \macro{var} et \macro{stddev}.
    \end{itemize}
    
    \separation
% ------------------------ %

    \medskip
    \item[2020-07-10] Première version \verb+0.0.0-beta+.
% ------------------------ %

% Changes shown - END 
\end{description}

\end{document}
