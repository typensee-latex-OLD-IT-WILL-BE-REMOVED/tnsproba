Nouvelle version mineure \verb+0.10.0-beta+.

\begin{itemize}[itemsep=.5em]
    \item \topic{Généralités}

    \begin{itemize}[itemsep=.5em]
        \item $\mathrm{P}$ est le nom par défaut d'une probabilité.

		\item Les noms des probabilités, des espérances, des variances et des écarts-types utilisent tous une police droite via \macro{mathrm} en coulisse.
    \end{itemize}


% ---------------------- %


    \item \topic{Espérance d'une variable aléatoire finie}

    \begin{itemize}[itemsep=.5em]
        \item \macro{expval} est utilisée en coulisse pour rédiger les espérances dans les détails des calculs.
        
		\item L'ordre des produits dans le calcul numérique est le même que celui dans la somme formelle.
    \end{itemize}
\end{itemize}


\separation

