Nouvelle version mineure \verb+0.9.0-beta+.

\begin{itemize}[itemsep=.5em]
    \item \topic*{Calculs de l'espérance dans le cas fini}
    	  \macro{calcexpval} rend facile la définition d'une variable aléatoire finie avec la possibilité de détailler le calcul de son espérance.


% ---------------------- %


    \item \topic*{Arbre de probabilités}
    	  pas mal de changements dans l'API et quelques nouveautés.

    \begin{itemize}[itemsep=.5em]
        \item \verb#\begin{probatree}<hideall># \verb#...# \verb#\end{probatree}#  remplace l'usage de l'environnement \env{probatree*}.


        \item \verb#\begin{probatree}<showall># \verb#...# \verb#\end{probatree}# remplace l'usage de l'environnement \env{probatree**}.


        \item De nouvelles macros permettent d'agir sur le texte des noeuds.

        \begin{enumerate}
        	\item \macro{ptreeTextOf} renvoie le texte d'un noeud.

        	\item \macro{ptreeNodeColor} change les couleurs du texte et/ou du fond d'un noeud.

        	\item \macro{ptreeNodeNewText} change le texte en plus éventuellement  les couleurs du texte et/ou du fond d'un noeud.
        \end{enumerate}


        \item \macro{ptreeFocus} a évolué.

        \begin{enumerate}
        	\item \macro{ptreeFocus[frame = start]} remplace l'ancien \macro{ptreeFocus}.

        	\item \macro{ptreeFocus[frame = none]} remplace \macro{ptreeFocus**} qui n'existe plus.

        	\item \macro{ptreeFocus[frame = nostart]} est utilisé par défaut et remplace \macro{ptreeFocus*} qui n'existe plus. On obtient dans ce cas une mise en valeur encadrant tous les noeuds sauf le tout 1\ier{}.

        	\item Les clés optionnelles \verb#lcol#, \verb#tcol# et \verb#bcol# permettent de choisir la couleur des arrêtes et des cadres, celle du texte et enfin celle du fond.
        \end{enumerate}


        \item \macro{aptreeFocus} a évolué presque comme \macro{ptreeFocus} car pour la couleur seule \verb#lcol# est disponible.


        \item \verb#tcol# remplace l'ancien \verb#col# de \macro{ptreeComment} et \macro{aptreeComment}.

		\item \verb#lcol# remplace l'ancien \verb#col# de \macro{ptreeFrame} et \macro{aptreeFrame}.



    \end{itemize}
\end{itemize}


\separation
