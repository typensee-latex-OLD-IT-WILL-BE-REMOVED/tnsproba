\documentclass[12pt,a4paper]{article}

\makeatletter
	\input{../config/header[fr].sty}

	\usepackage{01-base}
\makeatother



\begin{document}

\section{Probabilité}

\subsection{Probabilité \og simple \fg}

\newparaexample{}

\begin{latexex}
$\proba{A}$
\end{latexex}


% ---------------------- %


\newparaexample{Choisir le nom de la probabilité}

\begin{latexex}
$\proba[P]{A}$
\end{latexex}


% ---------------------- %


\subsection{Fiches techniques}

\IDmacro{proba}{1}{1}

\IDoption{} le nom de la probabilité. La valeur par défaut est \verb+p+.

\IDarg{} l'ensemble dont on veut calculer la probabilité.


% ---------------------- %


\subsection{Probabilité conditionnelle}

\newparaexample{Les deux écritures classiques}

La 1\iere{} notation, qui est devenue standard, permet de comprendre l'ordre des arguments.
\begin{latexex}
 $\probacond {B}{A}
= \probacond*{B}{A}$
\end{latexex}


% ---------------------- %


\newparaexample{Obtenir la formule de définition}

Le suffixe \prefix{exp} est pour \whyprefix{exp}{and} soit \inenglish{développer}
\footnote{
	Pour ne pas alourdir l'utilisation de \macro{probacond}, il a été choisi d'utiliser un suffixe au lieu d'un système de multi-options.
}.

\begin{latexex}
 $\probacondexp {B}{A}
= \probacondexp*{B}{A}$
\end{latexex}


% ---------------------- %


\newparaexample{Choisir le nom de la probabilité}

\begin{latexex}
 $\probacond    [P]{B}{A}
= \probacond*   [P]{B}{A}
= \probacondexp*[P]{B}{A}
= \probacondexp [P]{B}{A}$
\end{latexex}


% ---------------------- %


\subsection{Fiches techniques}

\IDmacro{probacond   }{1}{2}

\IDmacro{probacond*  }{1}{2}

\extraspace

\IDmacro{probacondexp}{1}{2}

\IDmacro{probacondexp*}{1}{2}


\IDoption{} le nom de la probabilité. La valeur par défaut est \verb+p+.

\IDarg{1} l'ensemble qui donne la condition.

\IDarg{2} l'ensemble dont on veut calculer la probabilité.


% ---------------------- %


\subsection{Espérance}

\newparaexample{}

\prefix{expval} vient de \whyprefix{exp}{ected} \whyprefix{val}{ue} soit \inenglish{espérance}.
\begin{latexex}
$\expval{X}$
\end{latexex}


% ---------------------- %


\newparaexample{Choisir le nom de l'espérance}

\begin{latexex}
$\expval[E_1]{X}$
\end{latexex}


% ---------------------- %


\subsection{Fiches techniques}

\IDmacro{expval}{1}{1}

\IDoption{} le nom de la fonction espérance. La valeur par défaut est \verb+E+.

\IDarg{} la variable aléatoire dont on veut calculer l'espérance.


\end{document}
