\documentclass[12pt,a4paper]{article}

\makeatletter
    \input{../config/header[fr].sty}

    \usepackage{01-tree}
\makeatother


% == EXTRAS == %


\begin{document}

%\section{Arbres pondérés}

\section{Fiches techniques}

\IDenv[n]{probatree}

\IDenv[n]{probatree*}

\Content{} un arbre codé en utilisant la syntaxe supportée par le package \verb+forest+.

\extraspace

\IDkey{pweight}  pour écrire un poids sur le milieu d'une branche.

\IDkey{apweight} pour écrire un poids au-dessus le milieu d'une branche.

\IDkey{bpweight} pour écrire un poids en-dessous du milieu d'une branche.

\extraspace

\IDkey{pframe} pour encadrer un sous-arbre depuis un noeud vers toutes les feuilles de celui-ci.


\separation


\IDmacro{ptreeFrame}{1}{3} \hfill \mwhyprefix{p}{robabilty}

\IDoption{} la couleur au format TikZ. La valeur par défaut est \verb#blue#.

\IDargs{1..3} noms de la sous-racine (à gauche), du noeud final en haut (à droite) et du noeud final en bas (à droite). En fait l'ordre n'est pas important ici.


\separation


\IDmacro{ptreeComment}{1}{2}

\IDoption{} la couleur au format TikZ. La valeur par défaut est \verb#black#.

\IDarg{1} le nom de la feuille.

\IDarg{2} le texte du commentaire.


\separation


\IDmacro{ptreeFocus  }{1}{1}

\IDmacro{ptreeFocus* }{1}{1}

\IDmacro{ptreeFocus**}{1}{1}

\IDoption{} la couleur au format TikZ. La valeur par défaut est \verb#blue#.

\IDarg{} les noms des noeuds dans le bon ordre et séparés par des barres verticales \verb#|#.
\end{document}
