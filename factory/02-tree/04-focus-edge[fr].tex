\documentclass[12pt,a4paper]{article}

\makeatletter
    \input{../config/header[fr].sty}

    \usepackage{01-basic}
    \usepackage{04-focus-edge}
    \usepackage{05-auto-num}
\makeatother


% == EXTRAS == %


\begin{document}

%\section{Arbres pondérés}

\subsection{Mettre en valeur des chemins}

\newparaexample{Juste avec deux noeuds}

Il est relativement aisé de mettre en valeur un chemin particulier comme dans l'exemple ci-après qui est une simple démo. montrant les différences entre \macro{ptreeFocus},  \macro{ptreeFocus*} et \macro{ptreeFocus**}.
Notez que les noms des noeuds sont séparés par des barres verticales \verb#|# et  qu'il est possible d'utiliser des espaces pour améliorer la lisibilité du code. 

\inputlatexex{tikz/probatree-focus-edge.tkz}

Voici ce qu'il faut retenir.

\begin{enumerate}
	\item \macro{ptreeFocus} encadre tous les noeuds.

	\item \macro{ptreeFocus*} n'encadre pas le tout premier noeud \emph{(typiquement cela est utile pour un chemin partant de la racine de l'arbre si celle-ci n'est pas nommée comme on le fait très souvent)}.

	\item \macro{ptreeFocus**} n'encadre aucun des noeuds.
	
	\item La couleur peut être changée via l'argument optionnel en utilisant les couleurs de type TikZ. Par défaut le bleu est utilisé.
\end{enumerate}


\begin{remark}
	Le fonctionnement interne de \verb#forest# empêche une coloration automatique des noeuds.
	Si vous souhaitez obtenir cet effet, il faudra donc ajouter les couleurs à la main pour chaque noeud comme dans l'exemple qui suit.

	\inputlatexex{tikz/probatree-focus-edge-color-nodes.tkz}
\end{remark}


% ---------------------- %


\newparaexample{Plusieurs noeuds d'un coup}

Rien de bien compliqué à condition de bien respecter l'ordre de saisie des noeuds.

\inputlatexexflat{tikz/probatree-focus-edge-long-star.tkz}


Avec \macro{ptreeFocus} on obtient l'arbre suivant où le mini disque initial
\footnote{
	Ce disque est en fait un carré aux coins arrondis autour d'un texte vide.
}
n'est pas forcément souhaité.

\input{tikz/probatree-focus-edge-long-no-star.tkz}


Avec \macro{ptreeFocus**} on obtient l'arbre ci-dessous.

\input{tikz/probatree-focus-edge-long-star-star.tkz}


%% ---------------------- %
%
%
%\newparaexample{De la racine à un noeud}
%
%Comme il est très courant de vouloir indiquer un chemin de la racine à une feuille, cette fonctionnalité est proposée par le package via les macros \macro{aptreeFocusAll},  \macro{aptreeFocusAll*} et \macro{aptreeFocusAll**} qui prennent comme unique argument obligatoire le noeud final du chemin
%\footnote{
%	Rien n'interdit de prendre un noeud interne au lieu d'u racine même si cela ne semble pas très utile a priori.
%}
%indiqué via le nom automatique fourni par \verb#forest#.
%On peut alors utiliser le code suivant très rapide à taper : voir la section \ref{tnsproba-autonum-forest} page \pageref{tnsproba-autonum-forest} où est expliqué d'où vient le nom \verb#!xxxxx# utilisé ci-après.
%

%\inputlatexexflat{tikz/probatree-focus-edge-all.tkz}
%\splitit{1312}

\end{document}
