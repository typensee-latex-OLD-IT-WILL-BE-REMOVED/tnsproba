\documentclass[12pt,a4paper]{article}

\makeatletter
    \input{../config/header[fr].sty}
    % == PACKAGES USED == %

\RequirePackage{amsmath}


% == DEFINITIONS == %

% Semantic probability

\newcommand\proba[2][p]{%
    \mathchoice{% * Display style
        #1\mskip-.65\medmuskip\left( #2 \right)%
    }{%           * Text style
        #1\mskip-.65\medmuskip\left( #2 \right)%
    }{%           * Script style
        #1\left( #2 \right)%
    }{%           * Script script style
        #1\left( #2 \right)%
    }
}


% Conditional probability

\newcommand\tnsproba@abstract@proba@cond[4]{%
    #1{\proba[#2]{#3 \cap #4}}{\proba[#2]{#4}}%
}


\newcommand\probacond{\@ifstar{\tnsproba@proba@cond@star}{\tnsproba@proba@cond@no@star}}

\newcommand\tnsproba@proba@cond@no@star[3][p]{%
    \proba[#1_{#2}]{#3}%
}

\newcommand\tnsproba@proba@cond@star[3][p]{%
    \proba[#1]{#3 \mid #2}%
}


\newcommand\eprobacond{\@ifstar{\tnsproba@proba@cond@exp@star}{\tnsproba@proba@cond@exp@no@star}}

\newcommand\tnsproba@proba@cond@exp@star[3][p]{%
    \tnsproba@abstract@proba@cond{\frac}{#1}{#3}{#2}
}

\newcommand\tnsproba@proba@cond@exp@no@star[3][p]{%
    \tnsproba@abstract@proba@cond{\dfrac}{#1}{#3}{#2}
}


% "Not" event

\newcommand\nevent[1]{%
    \overline{\kern.15ex#1\vphantom{#1^{x}}\kern.15ex}%
}


% Expected value - Variance - Standard deviation

\newcommand\expval[2][\mathrm{E}]{%
    \proba[#1]{#2}%
}

\newcommand\var[2][\mathrm{V}]{%
    \proba[#1]{#2}%
}

\newcommand\stddev[2][\sigma]{%
    \proba[#1]{#2}%
}



    \usepackage{01-basic}
    \usepackage{02-coment}
    \usepackage{03-frame}
    \usepackage{04-focus-edge}
    \usepackage{05-auto-num}
\makeatother


% == EXTRAS == %


\begin{document}

%\section{Arbres pondérés}

\subsection{Utiliser les noms automatiques donnés par \texttt{forest}} \label{tnsproba-autonum-forest}

Voyons comment obtenir le résultat suivant en indiquant tous les noeuds via les noms automatiques fabriqués par \verb@forest@.
%\footnote{
%	Le même rendu peut se taper bien plus efficacement via la macro à un argument \macro{ptreeFocusAll*} comme cela est expliqué dans la section précédente.
%}.

\inputlatexexcodeafter{tikz/probatree-autonum.tkz}


Voici comment s'y prendre.

\begin{enumerate}
	\item On utilise \macro{aptreeFocus} au lieu de \macro{ptreeFocus} où le préfixe \prefix{a} est pour \whyprefix{a}{uto}.
	      De façon similaire il existe les macros  \macro{aptreeComment} et  \macro{aptreeFrame}.


	\item Chaque nom automatique fabriqué par \verb#forest# commence par \texttt{!} . Ce caractère spécial n'est pas à indiquer car il sera ajouté automatiquement en coulisse.

	
	\item La racine est nommée \texttt{!} par \verb#forest# d'où le \verb#|# seul au début de l'argument de \macro{aptreeFocus*} ci-dessus afin d'indiquer un texte vide comme tout premier noeud.

	
	\item Pour voir ce qu'il faut faire pour un noeud autre que la racine, considérons par exemple \texttt{1321}. On indique en fait le chemin à suivre après la racine pour arriver au noeud voulu.
	\begin{itemize}
		\item Aller d'abord au \fbox{1}\,\ier{} noeud du niveau 1 qui ici est $A$.

		\item Aller ensuite au \fbox{3}\,\ieme{} noeud du niveau 2 qui ici est $D$.

		\item Aller après au \fbox{2}\,\ieme{} noeud du niveau 3 qui ici est $F$.
		
		\item Aller enfin au \fbox{1}\,\ier{} noeud du niveau 4 qui ici est $G$. C'est notre noeud nommé \fbox{\texttt{1321}}.
	\end{itemize}
\end{enumerate}


\begin{remark}
	Utilisez de préférence les noms automatiques car cela facilitera la maintenance de vos arbres sur le long terme.
	Si on reprend le tout premier exemple d'arbre décoré, il est bien plus simple de faire comme suit car on ne touche pas à la structure minimale du code de l'arbre.
	A titre illustratif on a utilisé \macro{ptreeFrame} au lieu de la clé \verb#pframe# directement dans l'arbre.
	
	\inputlatexexflat{tikz/probatree-showcase-autonum.tkz}
\end{remark}


\end{document}
