\documentclass[12pt,a4paper]{article}

\makeatletter
    \input{../config/header[fr].sty}

    \usepackage{01-tree}
\makeatother


% == EXTRAS == %


\begin{document}

%\section{Arbres pondérés}

\section{Fiches techniques}

\IDenv[n]{probatree}

\IDenv[n]{probatree*}

\Content{} un arbre codé en utilisant la syntaxe supportée par le package \verb+forest+.

\extraspace

\IDkey{pweight}  pour écrire un poids sur le milieu d'une branche.

\IDkey{apweight} pour écrire un poids au-dessus le milieu d'une branche.

\IDkey{bpweight} pour écrire un poids en-dessous du milieu d'une branche.

\extraspace

\IDkey{pcomment} pour ajouter un commentaire à la droite d'une feuille en utilisant le même alignement horizontal pour tous les commentaires.

\extraspace

\IDkey{pframe} pour encadrer un sous-arbre depuis un noeud vers toutes les feuilles de celui-ci.


\separation


\IDmacro{ptreeFrame}{1}{3} \hfill \mwhyprefix{p}{robabilty}

\IDoption{} la couleur au format TikZ. La valeur par défaut est \verb#blue#.

\IDargs{1..3} noms de la sous-racine (à gauche), du noeud final en haut (à droite) et du noeud final en bas (à droite). Ici l'ordre n'est pas important.

\end{document}
