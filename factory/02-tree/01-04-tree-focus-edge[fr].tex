\documentclass[12pt,a4paper]{article}

\makeatletter
    \input{../config/header[fr].sty}

    \usepackage{01-tree}
\makeatother


% == EXTRAS == %


\begin{document}

%\section{Arbres pondérés}

\subsection{Mettre en valeur des chemins}

Il est relativement aisé de mettre en valeur un chemin particulier comme dans l'exemple ci-après qui est une simple démo. montrant les différences entre \macro{ptreeFocus},  \macro{ptreeFocus*} et \macro{ptreeFocus**}.

\inputlatexex{tikz/probatree-focus-edge.tkz}

Voici ce qu'il faut retenir.

\begin{enumerate}
	\item \macro{ptreeFocus} encadre les noeuds initial et final.

	\item \macro{ptreeFocus*} encadre juste le noeud final.

	\item \macro{ptreeFocus**} n'encadre aucun des deux noeuds.
	
	\item La couleur peut être changée via l'argument optionnel. Par défaut le bleu est utilisé.
\end{enumerate}

\end{document}
