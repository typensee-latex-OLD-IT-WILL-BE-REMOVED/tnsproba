\documentclass[12pt,a4paper]{article}

\makeatletter
    \input{../config/header[fr].sty}
    % == PACKAGES USED == %

\RequirePackage{amsmath}


% == DEFINITIONS == %

% Semantic probability

\newcommand\proba[2][p]{%
    \mathchoice{% * Display style
        #1\mskip-.65\medmuskip\left( #2 \right)%
    }{%           * Text style
        #1\mskip-.65\medmuskip\left( #2 \right)%
    }{%           * Script style
        #1\left( #2 \right)%
    }{%           * Script script style
        #1\left( #2 \right)%
    }
}


% Conditional probability

\newcommand\tnsproba@abstract@proba@cond[4]{%
    #1{\proba[#2]{#3 \cap #4}}{\proba[#2]{#4}}%
}


\newcommand\probacond{\@ifstar{\tnsproba@proba@cond@star}{\tnsproba@proba@cond@no@star}}

\newcommand\tnsproba@proba@cond@no@star[3][p]{%
    \proba[#1_{#2}]{#3}%
}

\newcommand\tnsproba@proba@cond@star[3][p]{%
    \proba[#1]{#3 \mid #2}%
}


\newcommand\eprobacond{\@ifstar{\tnsproba@proba@cond@exp@star}{\tnsproba@proba@cond@exp@no@star}}

\newcommand\tnsproba@proba@cond@exp@star[3][p]{%
    \tnsproba@abstract@proba@cond{\frac}{#1}{#3}{#2}
}

\newcommand\tnsproba@proba@cond@exp@no@star[3][p]{%
    \tnsproba@abstract@proba@cond{\dfrac}{#1}{#3}{#2}
}


% "Not" event

\newcommand\nevent[1]{%
    \overline{\kern.15ex#1\vphantom{#1^{x}}\kern.15ex}%
}


% Expected value - Variance - Standard deviation

\newcommand\expval[2][\mathrm{E}]{%
    \proba[#1]{#2}%
}

\newcommand\var[2][\mathrm{V}]{%
    \proba[#1]{#2}%
}

\newcommand\stddev[2][\sigma]{%
    \proba[#1]{#2}%
}



    \usepackage{01-tree}
\makeatother


% == EXTRAS == %


\begin{document}

\section{Arbres pondérés}

\subsection{Au commencement était la forêt...}

Le gros du travail est fait par le package \verb+forest+ qui s'appuie \verb+TikZ+ dont on peut utiliser toute la machinerie afin d'obtenir des choses sympathiques comme ci-dessous et ceci à moindre coût neuronal comme vont le montrer les explications données dans les sections suivantes.

\inputlatexexcodeafter{TikZ/probatree-showcase.tkz}


\begin{remark}
	Jusqu'à la section \ref{tnsproba-autonum-forest} page \pageref{tnsproba-autonum-forest}, nous nommerons à la main les noeuds des arbres via \verb#name = ...# lorsque cela sera nécessaire.
	Dans la section indiquée nous verrons comment utiliser les noms automatiques donnés par le package \verb#forest#.
\end{remark}



% ---------------------- %


\subsection{Les bases}

\newparaexample{Le cas type}

Commençons par un arbre nu pour voir comment utiliser l'environnement \verb+probatree+ qui s'appuie en coulisse sur celui nommé \verb+forest+ du package éponyme.
L'exemple qui suit utilise juste les réglages spécifiques de mise en forme de l'arbre qui sont propres à \verb+probatree+.

\inputlatexex{TikZ/probatree-undecorated.tkz}


% ---------------------- %


\newparaexample{Ajouter des pondérations}

Dans le code suivant, ce sont les clés
\footnote{
	En fait du point de vue de TikZ, ce sont des styles.
}
\verb+pweight+, \verb+apweight+ et \verb+bpweight+ qui facilitent l'écriture des pondérations sur les branches.
Indiquons que \prefix{pweight} vient de \whyprefix{p}{robability} et \prefix{weight} soit \inenglish*{probabilité} et \inenglish{poids}.
Quant au \prefix{a} et au \prefix{b} au début de \prefix{apweight} et \prefix{bpweight} respectivement, ils viennent de \whyprefix{a}{bove} et \whyprefix{b}{elow} soit \inenglish*{dessus} et \inenglish{dessous}.


\inputlatexex{TikZ/probatree-weight-placement.tkz}


% ---------------------- %


\newparaexample{Des poids cachés partout}

On peut cacher tous les poids via l'environnement étoilé \verb+probatree*+ sans avoir à les effacer partout dans le code \LaTeX{} \emph{(ceci peut être utile lors de la rédaction d'exercices)}.

\inputlatexex{TikZ/probatree-weight-hide-all.tkz}


% ---------------------- %


\newparaexample{Des poids cachés localement}

Pour ne cacher que certains poids afin de produire par exemple un arbre à compléter, il faudra utiliser localement le style \verb+pweight*+ comme dans l'exemple ci-dessous \emph{(ceci aussi peut servir à rédiger des exercices)}.

\inputlatexex{TikZ/probatree-weight-hide-locally.tkz}


%% ---------------------- %
%
%
%\newparaexample{Un signe $=$ et/ou une virgule dans les étiquettes}
%
%Vous ne pouvez pas utiliser directement un signe $=$ ou une virgule dans les étiquettes des branches
%\footnote{
%	Ces deux symboles font partie de la syntaxe TikZ.
%}.
%Pour contourner cette limitation, il suffit de mettre le contenu de l'étiquette entre des accolades.
%
%\inputlatexex{TikZ/probatree-weight-coma-equal.tkz}
%
\end{document}
