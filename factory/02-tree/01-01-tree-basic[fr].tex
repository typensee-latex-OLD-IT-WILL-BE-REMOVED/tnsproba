\documentclass[12pt,a4paper]{article}

\makeatletter
    \input{../config/header[fr].sty}

    \usepackage{01-tree}
\makeatother


% == EXTRAS == %


\begin{document}

\section{Arbres pondérés}

\subsection{Que se passe-t-il en coulisse ?}

Le gros du travail est fait par le package \verb+forest+ qui utilise \verb+TiKz+. On peut donc faire appel à la machinerie de ce dernier et obtenir des choses sympathiques comme ci-dessous à moindre coût neuronal.

\inputlatexexcodeafter{tikz/probatree-showcase.tkz}


% ---------------------- %


\subsection{Les bases}

\newparaexample{Le cas type}

Commençons par un arbre nu pour voir comment utiliser l'environnement \verb+probatree+ qui s'appuie en coulisse sur celui nommé \verb+forest+ du package éponyme.
L'exemple qui suit utilise juste les réglages spécifiques de mise en forme de l'arbre qui sont propres à \verb+probatree+.

\inputlatexex{tikz/probatree-undecorated.tkz}


% ---------------------- %


\newparaexample{Ajouter des pondérations}

Dans le code suivant, ce sont les styles spéciaux supplémentaires \verb+pweight+, \verb+apweight+ et \verb+bpweight+ qui facilitent l'écriture des pondérations sur les branches
\footnote{
    \prefix{pweight} vient de \whyprefix{p}{robability} et \prefix{weight} soit \inenglish*{probabilité} et \inenglish{poids}.
    Quant au \prefix{a} et au \prefix{b} au début de \prefix{apweight} et \prefix{bpweight} respectivement, ils viennent de \whyprefix{a}{bove} et \whyprefix{b}{elow} soit \inenglish*{dessus} et \inenglish{dessous}.
}.

\inputlatexex{tikz/probatree-weight-placement.tkz}


% ---------------------- %


\newparaexample{Des poids cachés partout}

On peut cacher tous les poids via l'environnement étoilé \verb+probatree*+ sans avoir à les effacer partout dans le code \LaTeX{} \emph{(ceci peut être utile lors de la rédaction d'exercices)}.

\inputlatexex{tikz/probatree-weight-hide-all.tkz}


% ---------------------- %


\newparaexample{Des poids cachés localement}

Pour ne cacher que certains poids afin de produire par exemple un arbre à compléter, il faudra utiliser localement le style \verb+pweight*+ comme dans l'exemple ci-dessous.

\inputlatexex{tikz/probatree-weight-hide-locally.tkz}


% ---------------------- %


\newparaexample{Un signe $=$ et/ou une virgule dans les étiquettes}

Vous ne pouvez pas utiliser directement un signe $=$ ou une virgule dans les étiquettes des branches
\footnote{
	Ces deux symboles font partie de la syntaxe TikZ.
}.
Pour contourner cette limitation, il suffit de mettre le contenu de l'étiquette entre des accolades.

\inputlatexex{tikz/probatree-weight-coma-equal.tkz}

\end{document}
