\documentclass[12pt,a4paper]{article}

\makeatletter
    \input{../config/header[fr].sty}
    % == PACKAGES USED == %

\RequirePackage{amsmath}


% == DEFINITIONS == %

% Semantic probability

\newcommand\proba[2][p]{%
    \mathchoice{% * Display style
        #1\mskip-.65\medmuskip\left( #2 \right)%
    }{%           * Text style
        #1\mskip-.65\medmuskip\left( #2 \right)%
    }{%           * Script style
        #1\left( #2 \right)%
    }{%           * Script script style
        #1\left( #2 \right)%
    }
}


% Conditional probability

\newcommand\tnsproba@abstract@proba@cond[4]{%
    #1{\proba[#2]{#3 \cap #4}}{\proba[#2]{#4}}%
}


\newcommand\probacond{\@ifstar{\tnsproba@proba@cond@star}{\tnsproba@proba@cond@no@star}}

\newcommand\tnsproba@proba@cond@no@star[3][p]{%
    \proba[#1_{#2}]{#3}%
}

\newcommand\tnsproba@proba@cond@star[3][p]{%
    \proba[#1]{#3 \mid #2}%
}


\newcommand\eprobacond{\@ifstar{\tnsproba@proba@cond@exp@star}{\tnsproba@proba@cond@exp@no@star}}

\newcommand\tnsproba@proba@cond@exp@star[3][p]{%
    \tnsproba@abstract@proba@cond{\frac}{#1}{#3}{#2}
}

\newcommand\tnsproba@proba@cond@exp@no@star[3][p]{%
    \tnsproba@abstract@proba@cond{\dfrac}{#1}{#3}{#2}
}


% "Not" event

\newcommand\nevent[1]{%
    \overline{\kern.15ex#1\vphantom{#1^{x}}\kern.15ex}%
}


% Expected value - Variance - Standard deviation

\newcommand\expval[2][\mathrm{E}]{%
    \proba[#1]{#2}%
}

\newcommand\var[2][\mathrm{V}]{%
    \proba[#1]{#2}%
}

\newcommand\stddev[2][\sigma]{%
    \proba[#1]{#2}%
}



    \usepackage{01-tree}
\makeatother


% == EXTRAS == %


\begin{document}

%\section{Arbres pondérés}

\subsection{Commenter les racines}

\newparaexample{Tout aligner}

Que ce soit pour expliquer un arbre de probabilité, ou bien pour raisonner sur ce dernier, l'effet suivant est très utile
\footnote{
	Le package \texttt{forest} permet d'indiquer directement des mises en forme dans le code de l'arbre.
	L'auteur du présent package trouve bien plus efficace à l'usage de ne pas toucher au code minimal d'un arbre.
	Ceci explique donc le choix retenu de donner les décorations supplémentaires après le code de l'arbre.
}.


\inputlatexexflat{tikz/probatree-comment.tkz}


\begin{remark}
	Commenter un noeud interne ne provoquera pas d'erreur même si \macro{ptreeComment} n'a pas été conçu pour ceci.
	Ceci a été utilisé dans l'exemple d'introduction mais ça reste un petit hack.
\end{remark}



\newparaexample{Coller au plus près}

En utilisant \macro{ptreeComment*} au lieu de \macro{ptreeComment}, les commentaires seront proches des noeuds et donc non alignés verticalement. 
Avec l'exemple précédent on obtient la mise en forme qui suit.

\medskip

\input{tikz/probatree-comment-leftmost.tkz}

\end{document}


