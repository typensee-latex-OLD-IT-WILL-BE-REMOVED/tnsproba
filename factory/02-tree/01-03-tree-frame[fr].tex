\documentclass[12pt,a4paper]{article}

\makeatletter
    \input{../config/header[fr].sty}

    \usepackage{01-tree}
\makeatother


% == EXTRAS == %


\begin{document}

%\section{Arbres pondérés}

\subsection{Avec des cadres}

\newparaexample{Des cadres finaux}

Via la clé \verb#pframe# il est très aisé d'encadrer un sous-arbre final
\footnote{ 
	Un sous-arbre sera dit final si toutes ses feuilles correspondent à des feuilles de l'arbre initial. 
}
comme le montre l'exemple suivant
\footnote{ 
	Ce type de cadre est très utile d'un point de vue pédagogique. 
}.
Dans l'exemple ci-après nous utilisons la bidouille \verb+{},s sep = 1.3cm+ qui évite que les cadres se superposent.

\inputlatexex{tikz/probatree-frame-final.tkz}


\begin{remark}
	La clé \verb#pframe# est un cas particulier car tous les autres décorations se font en dehors de la définition de l'arbre
\end{remark}


% ---------------------- %


\newparaexample{Des cadres non finaux}

La macro \macro{ptreeFrame} permet facilement d'encadrer un sous-arbre non final.
Ceci nécessite d'utiliser des noms de noeuds.
Voici un exemple où la macro \macro{ptreeFrame} attend les noms de la racine et des deux noeuds finaux le plus haut et le plus bas.

\inputlatexex{tikz/probatree-frame-not-final.tkz}

\end{document}
